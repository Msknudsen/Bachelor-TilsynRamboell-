\chapter{Applikation}
I følgende afsnit gøres der rede for beslutningerne i forhold til udvikling af applikationens styresystem.

\section{Styresystem}
Der findes mange forskellige styresystemer i dag. De to brugte er dog iOS og Android. \\
IOS er Apples eget styresystem som er udviklet i forhold til deres mobile og tablet devices. \\
Android er et stort open source project som bliver brugt på ca. 80\% af alle mobile og tablet devices i dag. Der findes forskellige versioner af Android styresystemet alt efter hvilken producent der har leveret devices. \\
De mest kendte Android device leverendører er firmaer som Samsung, Huawei og HTC. De har alle en Android Core i deres styresystemer, men har alle også videre udviklet styresystemet, så det passer specifikt til deres devices. \\

Der er fordele og ulemper ved begge styresystemer, som tabellen herunder viser. \\


\begin{table} [H]
	\centering
	\begin{tabular}{ | m{10em} | m{10em}| m{10em} | } 
		\hline
		& iOS & Android   \\ 
		\hline
		Programmeringssprog  & Swift (Objective-C) & Java \\ 
		\hline
		Udviklingsplatform & Mac OS & Windows, Mac OS \\ 
		\hline
		Enheder & & \\
		\hline
	\end{tabular}
	\caption{Sammenligning af iOS og Android}
	\label{StyresystemTabel}
\end{table}


\section{Xarmarin}





