\chapter{Applikation}                                   
I følgende afsnit gøres der rede for beslutningerne i forhold til udvikling af applikationens styresystem.

\section{Styresystem}
Der findes mange forskellige styresystemer i dag. De to brugte er dog iOS \cite{iOS} og Android. \cite{Android} \\
IOS er Apples eget styresystem som er udviklet i forhold til deres mobile og tablet devices. \\
Android er et stort open source project som bliver brugt på ca. 80\% af alle mobile og tablet devices i dag. Der findes forskellige versioner af Android styresystemet alt efter hvilken producent der har leveret devices. \\
De mest kendte Android device leverendører er firmaer som Samsung, Huawei og HTC. De har alle en Android Core i deres styresystemer, men har alle også videre udviklet styresystemet, så det passer specifikt til deres devices.

Rambøll ønskede en applikation udviklet til iOS. Til en workshop med Rambøll, fandt vi ud af at der også var ansatte som bruger Android. Derfor blev der aftalt at udvikle cross-platform. \\

\section{Cross-platform udviklingsværktøjer}
I denne sektion vil der være en kort beskrivelse af nogle cross-platform udviklingsværktøjer.
\subsection{Xarmarin}

\subsection{PhoneGap}
PhoneGap er en platform udviklet af Adobe\cite{Adobe}. Her bruges teknologier som HTML5\cite{HTML5}, JavaScripts\cite{JavaScript} og CSS\cite{CSS}. \\
Deres platform tilbyder nogle bonus features som:
\begin{itemize}[-]
	\item Build server
	\item Easy share af applikationen 
	\item Mulitple plug-ins
\end{itemize}

\subsection{Corona}

\subsection{Sencha}
Sencha er en platform som er baseret på web applikations udvikling. De bruger teknologier som ES6\cite{ES6}, HTML5\cite{HTML5}, JavaScripts\cite{JavaScript} og CSS\cite{CSS}. \\
Deres platform tilbyder en masse ekstra ved hjælp af Ext JS frameworket. Nogle af disse er f.eks:
\begin{itemize}[-]
	\item Pre-integrerede og testede UI komponenter
	\item Data visualisering
	\item Back-end Data package
	\item Layout Manager
\end{itemize}

\section{Xarmarin}
I denne sektion vil der komme en mere dybdegående beskrivelse af hvordan Xarmarin platformen er bygget op.