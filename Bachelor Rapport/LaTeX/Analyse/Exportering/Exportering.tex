\section{Eksportering}                                   
I følgende afsnit gøres der rede for beslutningerne i forhold til exportering af information fra applikationen.

\subsection{Data eksportering}
I følgende afsnit vil der være en beskrivelse af forskellige eksporterings muligheder fra Rambøll Tilsyns App.

\subsubsection{Adobe Acrobat Reader}
Portable Document Format\cite{PDF} er et filformat, der gør det muligt at udveksle dokumenter uafhængigt af software, hardware eller operativ system.
Via Adobe Acrobat Reader\cite{AdobeReader} kan man gratis arbejde med PDF filer. Programmet giver mulighed for at bl.a. at lave kommentar og markere områder af filen. Det er dog ikke muligt at ændre i teksten, når filen først er i PDF formattet.

\subsubsection{Microsoft Word}
Word\cite{Office} er et word processor\cite{WordProcessor} værktøj til at redigere i tekst dokumenter.
Dette koster en Microsoft Office pakke. Her har du mulighed for at kommentere, opsætte teksten og direkte redigering af tekst direkte, inden man f.eks. udskriver.

\subsubsection{Microsoft Excel}
Excel\cite{Office} er et regnearks værktøj til beregninger, tabeller og data analyse. 
Dette koster en Microsoft Office pakke. Her er der mulighed for at sortere data efter alfabetisk orden, redigere direkte i teksten som i Word.

\subsection{Valg af data eksportering}
Efter nogen overvejelse blev der enighed om at der i Rambøll Tilsyns App, vil blive eksporteret data til Excel. Dette er valgt da det giver en masse muligheder i forhold til analyse og sortering i forhold til det information der bliver trukket ud af Rambøll Tilsyns App. \\
Med valget af at eksportere til Excel, udnytter vi også de ting som Excel kan og skal ikke selv til at lave sorterings algoritmer. Derved kan vi også lægge vores fokus andre steder i udviklingen af Rambøll Tilsyns App.