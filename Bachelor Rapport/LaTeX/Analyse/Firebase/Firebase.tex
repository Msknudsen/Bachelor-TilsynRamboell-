\section{Back-end}
I følgende sektion gøres der rede for beslutningerne i forhold til udvikling af back-end, heriblandt database, authentication og storage.

\subsection{Database}
I følgende afsnit redegøres for de beslutninger der tages angående valg af database. SQL eller NoSQL.
Der er fordele og ulemper ved begge database typer, herunder nævnes nogle af forskelligene mellem disse to typer.

\subsubsection{Database modellen}
Ved en SQL database kræves der Schema, som definerer data modellen, og danner et billede af hvad og hvordan data skal struktureres også kaldet et table. Ved NoSQL som er "schema-less", er det op til brugeren at definere en data model. NoSQL er mere fleksibelt når det omhandler ændringer i data modellen, men dette ligger mere ansvar hos udvikleren, som skal modellere dataen optimalt. \\

\subsubsection{Queries\cite{Query}}
Når data skal hentes tilbyder SQL database JOIN\cite{JOIN}, hvor data fra flere tabler flettes sammen og returneres. Dette har man ikke i en NoSQL database. \\
For at få tilsvarende effekt skal NoSQL databaser query flere gange og sammensætte resultatet selv.

\subsubsection{Transactions\cite{Transactions}}
 SQL tilbyder transaction som er nyttigt når man arbejder med flere databaser. Transaction gør at man kan opdatere flere databaser med et functions kald, hvis man skal have transaction tilsvarende effekt på NoSQL skal man selv sørge for det i applikations kode. 

 
\subsubsection{Skalering}
Efterhånden som applikationen bliver brugt, vokser mængden af lagret data. Her skal der tages stilling til om man vil skalerer sin server op eller skalerer ud ved at tilføje en ekstra server. Problemmet ved at udvide sin server er at der kommer et tidspunkt, hvor det vil blive for dyrt at udvide serveren og man vil hellere skalere serveren ud. \\
Fordelen ved NoSQL kontra en SQL database er at det er lettere at skalere ud. Men
når man kigger på at det er et inhouse projekt til Rambøll, vil der være begrænset antal af brugere til systemet og mængden af data vokser derfor i et begrænset omfang. 
Derfor vil det ikke være noget reelt betydning for dette projekt.


\subsection{Valg af back-end udviklingsværktøj}
I følgende afsnit redegøres for de beslutninger der er taget angående valg af database. \\ \\
Firebase kræver næsten ingen setup at bruge, hvilket var et af de store plusser for hvilken teknologi der skulle bruges til Ramboell Tilsyns App. 
For at starte Firebase i Android og iOS, skal man have en google account og oprette et projekt i google. Derefter skal man blot tilføje firebase til ens app, og tilføje de bibloteker man har behov for.
Det giver simpel adgang til data, brugere og filer pågrund af de bibloteker der udstilles. 

 Firebase blev valgt fordi, at der er ikke behov for at sætte en server op, og da Rambøll ikke lægger en server tilrådighed, er dette til oplagt mulighed, men ved valget af en anden database betyder, at man skal sætte en server op, som databasen kan være i, eller vælge at Cloud hoste\cite{{Cloud}} sin database. 

En af ulemperne med firebase er deres pris der kan blive dyrt hvis applikationen bliver brugt meget, \cite{{FirebasePricing}}, men da det kun er beregnet som en inhouse applikation med ca. 10 bruger er dette acceptablet.

\subsection{Firebase}
Firebase\cite{Firebase} er en development platform til mobile applikationer, som Google står bag. Firebase tilbyder forskellige slags funktionaliter til både Android og iOS applikationer. Herunder beskrives disse produkter fra Firebase, som Ramboell Tilsyns App benytter sig af: \\

\textbf{Authentication\cite{FirebaseAuth}}
\begin{itemize}[-]
	\itemsep 0.3em 
	\item[] Firebase Authentication giver en række funktioner indenfor User Management, der gør det let at logge på, opret og log af. Firebase Authentication tilbyder en række forskellige muligheder for login fra mail eller et andet sociale medie, som Facebook, Twitter osv. \\
	Bruger data som E-mail, navn og crypteret password, vil blive gemt i Firebase, og derfor vil der ikke være behov for, at Rambøll Danmark A/S skal sørge for at opsætte server med en bruger database, forbindelsen op til serveren og derefter vedligeholdelse. 
	
\end{itemize}	
\textbf{Realtime Database\cite{FirebaseRealtimeDB}}
\begin{itemize}[-]
	\itemsep 0.3em 
	\item[]  Firebase Realtime Database er et NoSQL cloud database, der gemmer data'en som JSON\cite{JSON}, med mulighed for synkroniser data med alle som benytter Ramboell Tilsyns App, der har forbindelse til internet. Databasen er Cloud hosted, hvilket betyder at som udviklere vil det give et "serverless" miljø at arbejde med. Dette betyder at arbejde med opsætning og vedligeholdelse af server ikke vil være nødvendigt. \\
	En af de store fordele er at det er en realtime database, der synchroniser data automatisk. Når der modtages ændringer i data vil alle devices der er koblet til internet modtage notifikation om ændringen og kan begynde at hente det ned. 
\end{itemize}
\textbf{Firebase Storage\cite{FirebaseStorage}}
\begin{itemize}[-]
	\itemsep 0.3em 
	\item[] Tilsvarende Firebase Database er Cloud Storage tilregnet for filer som video, billeder eller andre tilsvarende data. I dette tilfælde bliver Firebase Storage benyttet til at gemme PDF objekter som brugeren kan hente og uploade. En vigtigt egenskab Storage har er, at den er bygget til at genoptage download af filer i tilfælde af ustabilt internet. Dette kan være tilfældet hos Rambøll hvis de skal ud til konstruktions arbejde på en motorvej, hvor forbindelse er ikke lige så optimalt som i byen. 
\end{itemize}


\clearpage