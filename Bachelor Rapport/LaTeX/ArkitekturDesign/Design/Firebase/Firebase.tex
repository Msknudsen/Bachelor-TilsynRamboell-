Rambøll Tilsyn benytter tre Firebase produkter for at fungere, Authentication(Auth), Realtime Database og Storage.\\

 Firebase Auth\cite{FirebaseAuth} bruges til håndtering brugere som login og oprettelse. Den indeholder data som e-mail, en hashet password og en unik identifier, der bliver skabt af Firebase Authentication, når brugeren bliver skabt. Firebase tilbyder en Console , som er en hjemmeside, der giver mulighed for at se og manipulere data og konfigure sin Firebase. På Figur \ref{fig:FirebaseAuthPNG} kan en oversigt med informationer over bruger ses. 
\begin{figure}[H] % (alternativt [H])
	\centering
	\includegraphics[height=5cm, width=15cm]{../ArkitekturDesign/Design/Firebase/FirebaseAuth.PNG}
	\caption{Oversigt over authentication data i Firebase Console}
	\label{fig:FirebaseAuthPNG}
\end{figure}

Firebase Realtime Database bliver i forbindelse med applikationen brugt som en til at indeholde og give et simpelt overblik over information der er gemt i databasen, som vist på Figur \ref{fig:FirebaseDBPNG}.  
 
\begin{figure}[H] % (alternativt [H])
	\centering
	\includegraphics[height=10cm, width=8cm]{../ArkitekturDesign/Design/Firebase/FirebaseDB.PNG}
	\caption{Oversigt over database data i Firebase Console}
	\label{fig:FirebaseDBPNG}
\end{figure}

På Figur \ref{fig:FirebaseDBPNG} ses hvordan applikationens database er struktureret som et træ, med en root node\cite{rootNode}, der hedder ramboellfcm, som derfra bliver opdelt i 3 noder:
\textbf{pdf}, \textbf{project} og \textbf{user}.\\

\textbf{Pdf}-node indeholder en liste af Guid \cite{GUID}, som er keys til en pdfnode, disse bruges til at tilgå de specifikke pdfnoder. Under hver specifik pdf node er der 4 key-value-pair\cite{KVP}. Disse keys fungerer, også som en reference til Firebase Storage, når det bliver nødvendigt at hente PDF filen ned lokalt.

Dette gør, at man kan se, hvilket projekter og pdf'er der findes i databasen og vente med at downloade PDF'en indtil man har brug for det. På den måde kan man få en oversigt over, hvilket projekter og pdf'er der er og kun hente de data man har behov for. 
\begin{itemize}
	\item \textbf{created}\\
	Giver en dato om, hvornår objektet er oprettet med en nøjagtighed til sekunder, der kan forhøjes, hvis det føles nødvendig.\\
	\item \textbf{metaId}\\
	En referrence til JSON-filen der er gemt i Firebase Storage\cite{FirebaseStorage}, som indeholder de objekter der skal tegnes på pdf'en. Disse bliver brugt, når der skal indlæses de objekter, der bliver oprettet på pdf'en.\\ 
	
	\item \textbf{name}\\
	Navnet på PDF'en, som er skrevet i klar tekst, der er indtastet af brugeren, da en guid vil ikke give meget mening for brugeren.\\
	
	\item \textbf{updated}\\
	En timestamp der fortæller hvornår JSON-filen, som meta-filen referrer til sidst er blevet ændret. Dette bruger applikationen til at sammenligne den lokale version af JSON-filen med den der ligger i Storage.   
\end{itemize}

	\textbf{Project-node} har den samme structur som \textbf{pdf-node} af samme grund. Der er istedet for 6 key-value-pair. 
	\begin{itemize}
		\item  \textbf{address}\\
		Værdien gemt i address vil være et vejnavn, der fortæller, hvor projektet befinder sig.\\
		\item  \textbf{buildOwner}\\
		For diverse projekter vil der være bygherre, her vil der være gemt kontakt informationer til bygherren, som; navn, e-mail og telefonnummer. \\ 
		\item  \textbf{customer}\\
		Denne key indeholder kontaktinformationerne for hvem der udføres tilsynsrapporter for.\\
		\item  \textbf{name}\\
		Værdien gemt i name vil være navnet på projeket.\\
		\item  \textbf{number}\\
		Rambøll bruger et projektnummer til deres projekter og dette sætter de selv som brugere, og derfor gemmes denne information her\\
		\item  \textbf{pdf}\\
		En Referrence til en pdf-node, samt en referrence til Firebase Storage så applikationen kan  downloade PDF'en når den først har brug for det.\\
	\end{itemize}

\textbf{User} node indeholder brugerdefineret data der tilhører brugerne af applikationen. Det id der bliver brugt som key til user stammer fra Firebase Authentication og er en reference til User guid som ses på Figur \ref{fig:FirebaseAuthPNG}. 
\begin{itemize}
	\item \textbf{alias} \\
	Her gemmes telefonnummeret til brugeren. \\
	\item \textbf{email} \\
	Brugerens e-mail vil blive gemt i både Firebase- Authentication og Storage, for at gøre det tilgænglig at tilgå det information.\\
	\item \textbf{firstName} \\
	Fornavnet på brugeren.\\
	\item \textbf{firstName} \\
	Efternavnet på brugeren.\\

\end{itemize}

\clearpage

