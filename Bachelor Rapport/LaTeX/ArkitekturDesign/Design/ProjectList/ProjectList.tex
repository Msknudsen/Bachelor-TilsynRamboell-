\subsection{Projekt Oversigt} \label{sec:ProjectList}
Dette afsnit indeholder en gennemgang af grafisk brugergrænseflade, design og implementering af 'Project List' viewet i Rambøll Tilsyn.

\subsubsection{Design}
På figur \ref{fig:ProjctListSekvens} ses sekvens diagrammet for 'Project List' viewet til Rambøll Tilsyn.
%\begin{figure}[H] % (alternativt [H])
%	\centering
%	\includegraphics[height=15cm, width=15cm]{../ArkitekturDesign/Design/Login/LoginSekvensDiagram}
%	\caption{Sekvensdiagram for Login i Rambøll Tilsyn.}
%	\label{fig:ProjctListSekvens}
%\end{figure}

\clearpage

\subsubsection{Grafisk brugergrænseflade}
I ProjectListViewet er der en oversigt over hvilke projekter der ligger i databasen, samt mulighed for øverst at tilføje projekt eller tilføj bruger. Se figur \ref{fig:ProjectListView}
\begin{figure}[H] % (alternativt [H])
	\centering
	\includegraphics[height=12cm, width=10cm]{../ArkitekturDesign/Design/ProjectList/ProjectList}
	\caption{Login viewet som det er implementeret i Rambøll Tilsyn.}
	\label{fig:ProjectListView}
\end{figure}

\clearpage

\subsubsection{Implementering}
I dette afsnit vil der blive beskrevet funktionaliteten for de vigtigste funktioner i koden tilhørende 'Login' viewet.

På figur \ref{fig:ProjectListViewController}, ses konstrukteren til ProjectViewListController.
\begin{figure}[H] % (alternativt [H])
	\centering
	\includegraphics[height=5cm, width=17cm]{../ArkitekturDesign/Design/ProjectList/ProjectListViewController}
	\caption{}
	\label{fig:ProjectListViewController}
\end{figure}
Det første er at der hentes en reference node med navn PDF fra Firebase, og fortæller at denne node skal holdes synkroniseret, i tilfælde af data ændring. \\
Efterfølgende initialiseres JSON filen, og synkrorinisere efterfølgende med Firebase.

På figur \ref{fig:JSONFile}, ses funktionen for InstaniateJsonFile().
\begin{figure}[H] % (alternativt [H])
	\centering
	\includegraphics[height=5cm, width=15cm]{../ArkitekturDesign/Design/ProjectList/JSONFile}
	\caption{}
	\label{fig:JSONFile}
\end{figure}
Der oprettes en JSON fil, hvis der ikke findes en i forvejen. Denne gemmes lokalt.\\
Hvis der findes en JSON fil, læses data fra filen.

\clearpage

På figur \ref{fig:SyncWithDB}, ses funktionen for SyncWithFirebaseDatabase().
\begin{figure}[H] % (alternativt [H])
	\centering
	\includegraphics[height=8cm, width=15cm]{../ArkitekturDesign/Design/ProjectList/SyncWithDB}
	\caption{}
	\label{fig:SyncWithDB}
\end{figure}
Noden som er oprettet i konstrukteren, som ses på figur \ref{fig:ProjectListViewController}, den vedhæftes til et event, som kaldes hver gang noden ændres. \\
Når eventet kaldes, læses data fra eventet og gemmer dette i JSON filen. Efterfølgende fortælles controlleren at viewet skal opdateres.

På figur \ref{fig:ViewDidLoad}, ses funktionen for ViewDidLoad().
\begin{figure}[H] % (alternativt [H])
	\centering
	\includegraphics[height=3cm, width=10cm]{../ArkitekturDesign/Design/ProjectList/ViewDidLoad}
	\caption{}
	\label{fig:ViewDidLoad}
\end{figure}
Her sættes ListViewets source til at være TableSource.

På figur \ref{fig:TableSource}, ses konstrukteren til TableSource.
\begin{figure}[H] % (alternativt [H])
	\centering
	\includegraphics[height=2cm, width=15cm]{../ArkitekturDesign/Design/ProjectList/TableSource}
	\caption{}
	\label{fig:TableSource}
\end{figure}
Der initialiseres hvilke værdier der skal fremvises og der er givet en reference til ProjctListViewControlleren.

På figur \ref{fig:RowSelection}, ses funktionen for RowSelected().
\begin{figure}[H] % (alternativt [H])
	\centering
	\includegraphics[height=8cm, width=17cm]{../ArkitekturDesign/Design/ProjectList/RowSelection}
	\caption{}
	\label{fig:RowSelection}
\end{figure}
Der er en switch case, case 0 giver bruger mulighed for at navigere til 'Opret projekt' siden. Case 1 navigere brugeren til 'Opret bruger' siden. \\
Ellers skal den navigere brugeren til PDFViewControlleren, for det valgte projekt.

\clearpage
