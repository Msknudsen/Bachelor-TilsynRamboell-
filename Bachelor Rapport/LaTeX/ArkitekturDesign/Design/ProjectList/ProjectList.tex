\subsection{Projekt Oversigt} \label{sec:ProjectList}
Dette afsnit indeholder en gennemgang af den grafiske brugergrænseflade, design og implementering af 'Project List' viewet i Rambøll Tilsyn.

\subsubsection{Design}
På Figur \ref{fig:ProjctListSekvens} ses sekvensdiagrammet for 'Project List' viewet til Rambøll Tilsyn.
\begin{figure}[H] % (alternativt [H])
	\centering
	\includegraphics[height=15cm, width=15cm]{../ArkitekturDesign/Design/ProjectList/ProjektListSekvensDiagram}
	\caption{Sekvensdiagram for 'Project List' i Rambøll Tilsyn.}
	\label{fig:ProjctListSekvens}
\end{figure}

\clearpage

\subsubsection{Grafisk brugergrænseflade}
I ProjectListViewet er der en oversigt over, hvilke projekter der ligger i databasen samt mulighed for øverst at tilføje projekt eller bruger. Se Figur \ref{fig:ProjectListView}
\begin{figure}[H] % (alternativt [H])
	\centering
	\includegraphics[height=12cm, width=10cm]{../ArkitekturDesign/Design/ProjectList/ProjectList}
	\caption{'Project List' viewet, som det er implementeret i Rambøll Tilsyn.}
	\label{fig:ProjectListView}
\end{figure}

\clearpage

\subsubsection{Implementering}
I dette afsnit vil der blive beskrevet funktionaliteten for de vigtigste funktioner i koden tilhørende 'Project List' viewet.

På Figur \ref{fig:ProjectListViewController} ses funktionen for ProjectListViewController().
\begin{figure}[H] % (alternativt [H])
	\centering
	\includegraphics[height=3cm, width=17cm]{../ArkitekturDesign/Design/ProjectList/ProjectListViewController}
	\caption{Kode snip - ProjectListViewController() fra ProjectListViewController.cs}
	\label{fig:ProjectListViewController}
\end{figure}
Det første er, at der hentes en reference node med navn PDF fra Firebase som fortæller, at denne node skal holdes synkroniseret i tilfælde af dataændringer. \\
Efterfølgende initialiseres JSON-filen som synkrorinisere efterfølgende med Firebase.

På Figur \ref{fig:JSONFile} ses funktionen for InstaniateJsonFile().
\begin{figure}[H] % (alternativt [H])
	\centering
	\includegraphics[height=5cm, width=15cm]{../ArkitekturDesign/Design/ProjectList/JSONFile}
	\caption{Kode snip - InstaniateJsonFile() fra ProjectListViewController.cs}
	\label{fig:JSONFile}
\end{figure}
Der oprettes en JSON-fil, hvis der ikke findes en i forvejen. Denne gemmes lokalt.\\
Hvis der findes en JSON-fil, læses data fra filen.

\clearpage

På Figur \ref{fig:SyncWithDB} ses funktionen for SyncWithFirebaseDatabase().
\begin{figure}[H] % (alternativt [H])
	\centering
	\includegraphics[height=8cm, width=15cm]{../ArkitekturDesign/Design/ProjectList/SyncWithDB}
	\caption{Kode snip - SyncWithFirebaseDatabase() fra ProjectListViewController.cs}
	\label{fig:SyncWithDB}
\end{figure}
Noden, der oprettes i konstrukteren, som ses på Figur \ref{fig:ProjectListViewController}, vedhæftes et event, som kaldes hver gang noden ændres. \\
Når eventet kaldes, læses data fra eventet som gemmer dette i JSON-filen. Efterfølgende fortælles controlleren at viewet skal opdateres.

På Figur \ref{fig:ViewDidLoad} ses funktionen for ViewDidLoad().
\begin{figure}[H] % (alternativt [H])
	\centering
	\includegraphics[height=5cm, width=10cm]{../ArkitekturDesign/Design/ProjectList/ViewDidLoad}
	\caption{Kode snip - ViewDidLoad() fra ProjectListViewController.cs}
	\label{fig:ViewDidLoad}
\end{figure}
Her sættes ListViewets source til at være TableSource.

\clearpage

På Figur \ref{fig:TableSource} ses funktionen for TableSource().
\begin{figure}[H] % (alternativt [H])
	\centering
	\includegraphics[height=2cm, width=15cm]{../ArkitekturDesign/Design/ProjectList/TableSource}
	\caption{Kode snip - TableSource() fra TableSource.cs}
	\label{fig:TableSource}
\end{figure}
Der initialiseres, hvilke værdier der skal fremvises og der er givet en reference til ProjctListViewControlleren.

På Figur \ref{fig:RowSelection} ses funktionen for RowSelected().
\begin{figure}[H] % (alternativt [H])
	\centering
	\includegraphics[height=8cm, width=17cm]{../ArkitekturDesign/Design/ProjectList/RowSelection}
	\caption{Kode snip - RowSelected() fra TableSource.cs}
	\label{fig:RowSelection}
\end{figure}
Der er en switch case, casen CreateProject, som giver brugeren mulighed for at navigere til 'Opret projekt' siden. Casen CreateUser navigere brugeren til 'Opret bruger' siden. \\
Alternativt skal den navigere brugeren til 'PDFViewControlleren', for det valgte projekt.

\clearpage
