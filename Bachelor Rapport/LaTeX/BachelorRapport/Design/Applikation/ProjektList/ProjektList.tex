\subsection{Projekt oversigt}
I dette afsnit vises design, brugergrænseflade, implementering og test for 'Project List' viewet. For den fulde dokumentation henvises til Arkitektur og Design dokumentationens afsnit \ref{Design-sec:Opretbruger}.

\subsubsection{Design}
På Figur \ref{fig:ProjctListSekvens} ses sekvensdiagrammet for 'Project List' viewet til Rambøll Tilsyn.
\begin{figure}[H] % (alternativt [H])
	\centering
	\includegraphics[height=15cm, width=12cm]{../ArkitekturDesign/Design/ProjectList/ProjektListSekvensDiagram}
	\caption{Sekvensdiagram for 'Project List' i Rambøll Tilsyn.}
	\label{fig:ProjctListSekvens}
\end{figure}

\clearpage

\subsubsection{Grafisk brugergrænseflade}
I 'Project List' viewet er der en oversigt over, hvilke projekter der ligger i databasen, samt mulighed for tilføje projekt eller bruger. Se Figur \ref{fig:ProjectListView}
\begin{figure}[H] % (alternativt [H])
	\centering
	\includegraphics[height=12cm, width=10cm]{Design/Applikation/ProjektList/ProjectList}
	\caption{'Project List' viewet, som det er implementeret i Rambøll Tilsyn.}
	\label{fig:ProjectListView}
\end{figure}

\subsubsection{Implementering}
Før listen af projekter bliver vist for brugeren, initialiserer 'Project List' viewet en forbindelse til Firebase, hvor den sætter et synkroniseringsevent, som kaldes såfremt, der sker ændringer i PDF-filerne. \\
Dernæst opretter den en JSON-fil, som indeholder alt projektinformation fra Firebase. \\
Controlleren opretter nu et nyt TableView som har sourcen TableSource. Her laver den en liste bestående af alle projekter og muligheden for at tilføje projekt og bruger. \\
For en mere deltaljeret beskrivelse af implementeringen og kode snips, henvises til Arkitektur og Design dokumentations afsnit \ref{Design-sec:ProjectList}.

\clearpage