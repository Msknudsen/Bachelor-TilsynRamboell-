\chapter*{Læsevejledning}
Denne rapport redegør for den proces som projektet har fulgt og skal give læseren et indblik i projektdokumentationen med vægt på projektets særegenheder. Rapporten følger som udgangspunkt strukturen, der er anbefalet i det vejledende dokument udleveret af IHA / ASE.\\
Da rapporten er en del af den samlede dokumentation for projektet, vil der forekomme
henvisninger til tekst og figurer i projekt dokumentationen.


Rapporten kan overordnet inddeles i 5 sektioner:\\
I første sektion gives der i Kapitel 1 en indledende beskrivelse af det valgte projekt og
dets afgrænsninger. I Kapitel 2 bliver kravene til systemet beskrevet.

I anden sektion beskrives der kort i Kapitel 3 de arbejdsgange og metoder, som er blevet brugt
under projektet. For en mere uddybende procesbeskrivelse, henvises der til procesrapporten.
I Kapitel 4 beskrives kort om den analyse, der er lavet til projektet.

I tredje sektion ses arkitekturen til Rambøll Tilsyn samt, hvordan det er blevet designet, implementeret og testet. 
Sektionen dækker over Kapitel 5, 6 og 7.

I fjerde sektion af rapporten findes resultater og diskussionen samt, hvilke overvejelser
der er gjort for at projektet kan videreudvikles. Fjerde sektion indeholder ligeledes generelle erfaringer. 
Dette sker i Kapitel 8, 9 og 10.

Femte sektion indbefatter Kapitel 11, hvor der konkluderes på projektet som helhed og
der drages paralleller til indledningen.

I slutningen af rapporten findes ordforklaring og litteraturliste.

%\clearpage