\chapter{Test}
Dette kapitel indeholder en kort beskrivelse af de tests som er blevet udført i forhold til validering af Rambøll Tilsyn. \\

\section{Automatiseret test}
I følgende afsnit beskrives unit test, der er lavet. Framworket Nunit\cite{NUnit} er blevet brugt til testing af klasser. Den fulde automatiserede test dokumentation, henvises i Modul- og Integrationstest dokumentet Afsnit \ref{Test-sec:Unit} Unit Test. \\

\textbf{Validator}
Validator-klassen er en statisk klasse, som håndterer validering af input-felterne i applikationen. Applikationens testcases ses på Figur \ref{fig:ValidatorUnit}.
\begin{figure}[H]
	\centering
	\includegraphics[width=0.6\linewidth]{Test/ValidatorUnit}
	\caption{Screenshot af test sessionen på ValidatorUnit.}
	\label{fig:ValidatorUnit}
\end{figure}

En af unit testene, som kan ses på Figur \ref{fig:ValidatorUnit}, er for funktionen PasswordIsValid. Her bliver der testet for både valide og invalide kodeord.\\
Funktionen PasswordIsValid kontrollerer at et password opfylder følgende krav:
\begin{itemize}
	\item Minimum 6 karakter lang
	\item Minimum et stort bogstav
	\item Minimum et lille bogstav
	\item Minimum et tal
	\item Speciel tegn er valgfrit
	\item Må ikke have mellemrum
\end{itemize}

Validator-klassen er den eneste klasse, der automatisk testes, da det er den eneste klasse, der indeholder en algoritme. Koden, som benyttes i prototypen, ligger i forskelllige frameworks og derfor tages der højde for at disse er vel implementeret og gennemtestet. Dette er grunden til, at der ikke implementeret mere automatisk test. 

\clearpage

\section{Manuel test}
Denne sektion indeholder de manuelle tests for 'Login' og 'Opret bruger' viewene. Der er skrevet tilsvarende test cases for alle de implementerede views. Disse kan findes i Modul- og Integrationstest dokumentets Afsnit \ref{Test-sec:ManuelTest} Manuelle Test. \\ \\
\textbf{Login tests} \\
\textbf{Pass/fail criteria:} \\
Alle steps skal verificeres på Rambøll Tilsyn applikationen:
\begin{itemize}[-]
	\item Verificer at bruger kan logge ind med korrekte brugeroplysninger.
	\item Verificer at bruger ikke kan logge ind med forkert e-mail.
	\item Verificer at bruger ikke kan logge ind med forkert kodeord. \\
\end{itemize}

\textbf{Opret bruger test} \\
\textbf{Pass/fail criteria:} \\
Alle steps skal verificeres på Rambøll Tilsyn applikationen:
\begin{itemize}[-]
	\item Verificer at en bruger kan oprettes, hvis alle felter er udfyldte korrekt.
	\item Verificer at en bruger ikke kan oprettes, hvis e-mail feltet mangler at blive udfyldt.
	\item Verificer at en bruger ikke kan oprettes, hvis kodeord feltet mangler at blive udfyldt.
	\item Verificer at en bruger ikke kan oprettes, hvis fornavn feltet mangler at blive udfyldt.
	\item Verificer at en bruger ikke kan oprettes, hvis efternavn feltetmangler at blive udfyldt.
	\item Verificer at en bruger ikke kan oprettes, hvis telefonnummer feltet mangler at blive udfyldt.
	\item Verificer at en e-mail skal indeholde et snabel-a (@) for at blive godkendt.
	\item Verificer at kodeordet skal overholde kravene omkring minimum 6 cifre, et stort bogstav, et lille bogstav og et tal.
\end{itemize}

De manuelle tests til Rambøll Tilsyn er med til at sikre at både funktionalitet og forløb i applikationen fungerer.