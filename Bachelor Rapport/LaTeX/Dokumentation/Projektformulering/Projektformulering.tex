
	\chapter{Projektformulering}
	
	Strøm Hansen er en landsdækkende automations- og installationsleverandør, som har ansvaret for trafikregulering i Randers Kommune. \\Dette indebærer blandt andet servicering af trafikregulering i vej- og gadekryds.
	Leverandøren af trafikreguleringsinstallationerne, som er Swarco/ITS-teknik anvender en server, hvori Strøm Hansen kan hente notifikationer om evt. slukket lyskryds via mail og SMS. Derudover er der mulighed for visuelt at se status på de forskellige lyskryds via et online anlægskort.
	
	Strøm Hansens krav, som de har fra Randers Kommune til den allerede etablererede ”log” lyder som nedenstående:
	
	\begin{itemize}[-]
		\item Randers kommune vil gerne kunne skrive bemærkninger ved de forskellige hændelser\\
		På nuværende tidspunkt har de kun læse adgang\\
		
		\item Søgning i alle hændelser ud fra dato\\
		
		\item Randers kommune ønsker en notifikation enten via. mail eller SMS når montøren opretter en ny log, for at få feedback, når opgaven er påbegyndt eller afsluttet.\\
		
		\item Anlægsdata ønskes synlig \\
		På nuværende tidspunkt er det ikke synlig for Randers kommune
	\end{itemize}
		
	Problemstillingen forstilles løst ved at have et centralt system, hvor data gemmes i egen database.
	
	Derudover udvikles applikationer til Strøm Hansens arbejdstelefoner samt PC'er.
	Denne skal notificere om fejlkoden (- eller fejlbeskrivelsen), hvorved medarbejdere har mulighed for at tjekke sig ind på pågældende opgave, hvilket andre medarbejdere kan se på applikationen. Det skal være muligt for medarbejderen at ændre fejlbeskrivelse, hvis vedkommende står over for et teknisk problem og har brug assistance. \\
	


