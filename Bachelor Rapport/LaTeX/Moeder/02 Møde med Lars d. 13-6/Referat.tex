\input{../../Preamble/Preamble}
\newcommand{\HRule}{\rule{\linewidth}{0.1mm}}

\begin{document}
	\begin{center}
		{\huge \bfseries \textsc{Mødereferat nr. 2}}\\
		\textsc{\large Bachelorprojekt - Gruppe ?}\\[0.3cm]
	\end{center}
	\begin{tabular}{ll}
	\large \textbf{Dato:} & 13/06/2017  	\\ % Sted
	\large \textbf{Tid:}  & 08:00-9:00 	\\ % Tid
	\large \textbf{Sted:} & Katrinebjerg		\\ % Lokation
	\large \textbf{Deltagere:} & Ao Li (AL), Morten Knudsen (MK), Lars Christian - Vejleder \\
	\large \textbf{Udeblevet:}
	\end{tabular}\\
	\phantom{\,}\hspace{0.1em} \large \textbf{Referat:}
	\begin{enumerate}
		\itemsep 0.3em 
		\item \textbf{Mødeleder}\\
			Morten Knudsen
		\item \textbf{Referent}\\
			Morten Knudsen
		\item \textbf{Godkendelse af sidste referat}
			
		\item\textbf{Opfølgning på aktionspunkter}

		\item \textbf{Update}
			\begin{itemize}[-]
			\item Kort briefing af Rambøll mødet til Lars\\
			Kort gennemgang af referatet fra mødet med Rambøll	\\
			
			\item Kravspec
			Lav mellem 5-10 usecases \\
			 - Lav use cases / stories ud fra de forskellige roller
			Aktør kontekst diagram
			Use case diagram
			Ikke funktionelle krav \\
			 - Kunde krav \\
			Domaine model (rigt billede) / system overblik \\
			Evt. aktitektur diagram / interface diagram \\
			Unikke ID til alle krav - CRS og TRS\\
						
			\item Fremtidig møder
			Forslag at mødes hver 7-14 dag med Lars \\
			- 60 timer sat af til Lars (ca. 2 uger pr uge) \\
			
			\item Udviklings værktøjer
			Cordova er også et cross platform udviklingsværktøj, til udvikling af mobil app i HTML, CSS og JS \\
			
			Positivt response fra Lars \\
			
			\item Github
			Godkendelse af Rambøll til at bruge github eller et andet git værktøj
			
			\item General krav skrivning
			Unikke \\
			Entydig \\
			Testbar \\
			Hvert krav indeholder indeholder to "shall", undgå og \\
			Hold os til hvad den skal kunne. Skriv hvad den IKKE kan i afgrænsnings afsnit \\
			
			\item Bekymringer
			Ændringer af registreringer, hvordan skal man kunne lave ændringer, på hvilke platforme ønsker man at lave ændringer, skal ændringer lægges tilbage på databasen?
		\end{itemize}
	
		\item \textbf{Tidspunkt for næste møde} \\
		Afholdes omkring projekt opstart (Evt. første uge i august) \\
		\item \textbf{Evt.}
		
		\item \textbf{Opgaver} \\
		\textbf{Ao}: \\
		Domaine model \\
		User stories \\
		Tilsvarende projekter \\
		Kunde krav (Ikke funktionelle krav) \\
		Xarmain guidelines \\
		
		\textbf{Morten}: \\
		Rambøll omkring github \\
		Projektstyring \\
		User stories \\
		Kunde krav (Ikke funktionelle krav) \\
		Arbejdstid og sted \\
		Mail Lene omkring gruppe nummer \\
		
		\textbf{Deadline}: \\
		14/6 kl. 12.00
	\end{enumerate}
\end{document}