\input{../../Preamble/Preamble}
\newcommand{\HRule}{\rule{\linewidth}{0.1mm}}

\begin{document}
	\begin{center}
		{\huge \bfseries \textsc{Mødereferat nr. 15}}\\
		\textsc{\large Bachelorprojekt - Projekt nr. 17103}\\[0.3cm]
	\end{center}
	\begin{tabular}{ll}
	\large \textbf{Dato:} 15/12/2017  	\\ % Sted
	\large \textbf{Tid:}  15:30-16:00 	\\ % Tid
	\large \textbf{Sted:} Skype		\\ % Lokation
	\large \textbf{Deltagere:} Ao Li (AL), Morten Knudsen (MK), Lars Christian - Vejleder \\
	\large \textbf{Udeblevet:} 
	\end{tabular}\\
	\phantom{\,}\hspace{0.1em} \large \textbf{Referat:}
	\begin{enumerate}
		\itemsep 0.3em 
		\item \textbf{Mødeleder}\\
			Morten Knudsen
		\item \textbf{Referent}\\
			Morten Knudsen

		\item \textbf{Update}
			\begin{itemize}[-]
				\item Rapport 
				\item Produkt			
			\end{itemize}
		
		\item \textbf{Analyse}
		\begin{itemize}[-]
			\item Ser meget bedre ud, efter Forms er blevet beskrevet.			
		\end{itemize}
					
		\item \textbf{Spørgsmål}
			\begin{itemize}[-]
				\item Mere i rapporten
				Tag Opret bruger med \\
				\item Test
				Negativ test \\
				Skrive hvorfor der ikke er mere unit test \\
				Tag en test case ud og beskriv den \\
				Manuel test dækker GUI test \\
				\item Accepttest
				Laver drive run med Lars mandag d. 18/12 \\	
				\item Fælles klasse diagram
				Smides i arkitekturen \\
				Dem vi har lavet og hvordan de er koblet \\
				Xamarin klasse diagram / beskrivelse af hvilke klasser som bruges\\
				\item Lav en generel reference til mockup i starten af arkitektur og design dokumentationen \\
					
			\end{itemize}
	
		\item \textbf{Tidspunkt for næste møde} \\
			Mandag d. 18/12 kl. 09.00 på KB \\
						
		\item \textbf{Evt.}
			\begin{itemize}[-]
				\item 
			\end{itemize}
			
	\end{enumerate}
\end{document}