\chapter{Konklusion}
Dette kapitel har til formål at beskrive gruppens samlede konklusion for projektets proces, og efterfulgt af en kort individuel proces fra gruppens to medlemmer. 

%\section{Konklusion på processen}
Generelt var processen bag projektet god, og gruppen arbejdede godt og effektivt sammen.
Der blev implementeret dele af scrum i processen for at prøve at arbejde mere struktureret gennem hele projektets forløb. Dette betød at gruppen havde et godt overblik gennem hele projektet. \\
Der skulle et par gange undervejs i forløbet genplanlægges, da gruppen stødte ind i nogle problemer med først byggemiljø og derefter begrænsninger i Xamarin Forms. Dette gjorde, at nogle milepæle blev flyttet i tidsplanen og den derved blev mere kompakt end først planlagt. \\
Men dette ændrede ikke det store i processen bag udviklingen, da gruppen fortsatte i samme spor som tidligere, med omrokerede arbejdsopgaver.

ASE-modellen er en metode som gruppen havde arbejdet med tidligere. Dette gav et godt værktøj at støtte sig op af, da arbejdsgangen var kendt fra tidligere projekter. \\
Modellen gav fra start et godt overblik over hele processen og sikrede, at dokumentationen blev udført undervejs i projektet.

Samarbejdet i gruppen har været yderst tilfredsstillende. Forventningerne til projektets forløb og resultat var afstemt fra starten, og alle i gruppen leverede en fremragende indsats.

%\section{Konklusion fra gruppens medlemmer}
%\subsection{Ao Li}

%\subsection{Morten Sand Knudsen}