\chapter{Konklusion}
Dette kapitel har til formål at beskrive gruppens samlede konklusion for projektets proces, efterfulgt af en kort individuel proces fra gruppens to medlemmer. 

%\section{Konklusion på processen}
Generelt var processen bag projektet god og gruppen arbejdede godt og effektivt sammen.
Der var implementeret dele af scrum i processen for at prøve at arbejde mere struktureret gennem hele projektets forløb. Dette gjorde at gruppen af et godt overblik gennem projektet. \\
Der skulle dog et par gange undervejs i projektet genplanlægges, da gruppen ramte ind i nogle problemer med først bygge miljø og derefter begrænsninger i Xamarin Forms. Dette gjorde at nogle milepæle skulle flyttes i tidsplanen og den derved blev mere kompakt end først planlagt. \\
Men dette ændrede ikke det store i processen bag udviklingen, da gruppen fortsatte i samme flow som tidligere, bare med nogle omrokerede arbejdsopgaver.

ASE-modellen er en metode som gruppen havde arbejdet med før. Dette gav et godt værktøj at støtte sig op af og at arbejdsgangen var kendt fra tidligere projekter. \\
Modellen gav fra start et godt overblik over hele processen og sikrede at dokumentationen blev udført undervejs i projektet.

Samarbejdet i gruppen var yderst tilfredsstillende. Forventningerne til projektets forløb og resultat var afstemt fra startm og alle i gruppen leverede en fremragende indsats.

%\section{Konklusion fra gruppens medlemmer}
%\input{Konklusion/subpages/Ao}
%\input{Konklusion/subpages/Morten}