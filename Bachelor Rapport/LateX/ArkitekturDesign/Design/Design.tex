\chapter{Design}
Dette kapitel indeholder en beskrivelse af design valgene for Firebase databasen og applikationen, samt hvordan disse er implementeret. \\

\section{Firebase database}
\section{Back-end}
I følgende sektion gøres der rede for beslutningerne i forhold til udvikling af back-end, heriblandt database, authentication og storage.

\subsection{Database}
I følgende afsnit redegøres for de beslutninger der tages angående valg af database. SQL\cite{SQL} eller NoSQL\cite{NoSQL}.
Der er fordele og ulemper ved begge database typer, herunder nævnes nogle af forskelligene mellem disse to typer.

\subsubsection{Database modellen}

Ved en SQL database kræves der Schema\cite{Schema}, som definerer data modellen, og danner et billede af hvad og hvordan data skal struktureres også kaldet et table. Ved NoSQL som er "schema-less", er det op til brugeren at definere en data model. NoSQL er mere fleksibelt når det omhandler ændringer i data modellen, men dette pålægger ansvar hos applikations udviklerne til, at modellere dataen optimalt.

\subsubsection{Queries\cite{Query}}
Når data skal hentes tilbyder SQL database JOIN\cite{JOIN}, hvor data fra flere tabler flettes sammen og returneres. Dette har man ikke i en NoSQL database. \\
For at få tilsvarende effekt skal NoSQL databaser query flere gange og sammensætte resultatet selv.

\subsubsection{Transactions\cite{Transactions}}
 SQL tilbyder transaction, hvilket er nyttigt, når der arbejdes med flere databaser, der skal opdateres samtidigt. Transaction tillader, at opdatere flere databaser under et funktionskald til serveren. For at opnå tilsvarende effekt på NoSQL databaser, skal dette implementeres applikations kode. 

\subsubsection{Skalering}
Efterhånden som applikationen bliver brugt, vokser mængden af lagret data. Her tages der stilling til om man vil skalerer sin server op eller skalerer ud ved at tilføje en ekstra server. Problemmet ved at udvide serveren er, at der kommer et tidspunkt, hvor det vil blive for dyrt at udvide serveren og man vil hellere skalere serveren ud. \\
Fordelen ved NoSQL iforhold til SQL er, at det er lettere at skalere ud. Men
når der kigges på, at det er et inhouse projekt til Rambøll, vil der være begrænset antal af brugerer til systemet , så mængden af data vokser derfor i et begrænset omfang. 
Derfor vil dette ikke være noget reelt betydning for dette projekt.

\subsection{Back-end udviklingsværktøjer}
I denne sektion vil der beskrives kort om nøglepunkterne om nogle af databaserne, der blev taget overvejelse. 

\subsubsection{MySQL\cite{MySQL}}
En populær SQL database, der benytter sig af SQL.
\begin{itemize}[-]
	\item ACID\cite{ACID} operationer understøttes
	\item Tilbyder mange query muligheder
\end{itemize}

\subsubsection{Firebase\cite{Firebase}}
Firebase indeholder flere databaser til hvert deres formål, som kan vælges efter behov. 
\begin{itemize}[-]
	\item Server løst miljø
	\item Databaser med SDK til både android og IOS
\end{itemize}

\subsubsection{MongoDB\cite{mongoDB}}
En dokument baseret database.
\begin{itemize}[-]
	\item Hurtigt og simpelt søgning
	\item Bygget til at kunne skalerer og understøtte fleksibilitet i data modeller.  \\
	
\end{itemize}

\subsection{Valg af back-end database}
I denne sektion vil der komme en mere dybdegående beskrivelse af hvordan Firebase er bygget op og hvorfor Firebase er valgt til dette projekt.\\

Firebase\cite{Firebase} er en development platform til mobile applikationer, som Google står bag. Firebase tilbyder forskellige slags funktionaliter til både Android og iOS applikationer. Herunder beskrives disse produkter fra Firebase, som Ramboell Tilsyn benytter sig af.

For at starte Firebase i Android og iOS, skal man have en google account og oprette et projekt i google. Derefter skal man blot tilføje firebase til ens app, og tilføje de bibloteker man har behov for.
Det giver simpel adgang til data, brugere og filer pågrund af de bibloteker der udstilles. 

Firebase database blev valgt fordi, at der er ikke behov for at sætte en server op, og da Rambøll ikke lægger en server tilrådighed, er dette til oplagt mulighed, men ved valget af en anden database betyder, at man skal sætte en server op, som databasen kan være i, eller vælge at Cloud hoste\cite{{Cloud}} sin database. 

Firebase kræver næsten ingen setup at bruge, hvilket var et af de fordele for hvilken teknologi der skulle bruges til Rambøll Tilsyn. Firebase tilbyder tre forskellige databaser med hvert deres formål. 

En af ulemperne med Firebase er deres pris der kan blive dyrt hvis applikationen bliver brugt meget. Men da det kun er beregnet som en inhouse applikation med ca. 10 bruger er dette acceptablet. For at vide mere om deres pris niveau kan i læse mere \cite{FirebasePricing}. 

\textbf{Authentication\cite{FirebaseAuth}}
\begin{itemize}[-]
	\itemsep 0.3em 
	\item[] Firebase Authentication giver en række funktioner indenfor User Management, der gør det let at logge på, opret og log af. Firebase Authentication tilbyder en række forskellige muligheder for login fra mail eller et andet sociale medie, som Facebook, Twitter osv. \\
	Bruger data som E-mail, navn og crypteret password, vil blive gemt i Firebase, og derfor vil der ikke være behov for, at Rambøll Danmark A/S skal sørge for at opsætte server med en bruger database, forbindelsen op til serveren og derefter vedligeholdelse. 
\end{itemize}	

\textbf{Realtime Database\cite{FirebaseRealtimeDB}}
\begin{itemize}[-]
	\itemsep 0.3em 
	\item[]  Firebase Realtime Database er et NoSQL database, der gemmer data'en som JSON\cite{JSON}, med mulighed for synkroniser data med alle som benytter Ramboell Tilsyn, der har forbindelse til internet. Databasen er cloud hosted, hvilket betyder at som udviklere vil det give et "serverless" miljø at arbejde med. Dette betyder at arbejde med opsætning og vedligeholdelse af server ikke vil være nødvendigt. \\ Hvor man kan tilgå databasens indhold via api kald. 
	En af de store fordele er at det er en realtime database, der synchroniser data automatisk. Når der modtages ændringer i data vil alle devices der er koblet til internet modtage notifikation om ændringen og kan begynde at hente det ned. 
\end{itemize}

\textbf{Firebase Storage\cite{FirebaseStorage}}
\begin{itemize}[-]
	\itemsep 0.3em 
	\item[] Tilsvarende Firebase Database er Cloud Storage tilregnet for filer som video, billeder eller andre tilsvarende data. I dette tilfælde bliver Firebase Storage benyttet til at gemme PDF objekter som brugeren kan hente og uploade. En vigtigt egenskab Storage har er, at den er bygget til at genoptage download af filer i tilfælde af ustabilt internet. Dette kan være tilfældet hos Rambøll hvis de skal ud til konstruktions arbejde på en motorvej, hvor forbindelse er ikke lige så optimalt som i byen. \\
\end{itemize}
\clearpage


\section{Applikation} \label{sec:ApplikationDesign}
Implementeringen af applikationen sker i Xamarin iOS, da der i analysen var blevet overset nogle begrænsninger i forhold til funktionaliteten når der skulle arbejdes med PDF'er i Xamarin Forms\cite{Forms}. \\ \\
Til udvikling af applikationen er der brugt et MVC design \cite{MVC}, som vist herunder, på figur \ref{fig:MVC}
\begin{figure}[H] % (alternativt [H])
	\centering
	\includegraphics[height=8cm, width=8cm]{../ArkitekturDesign/Design/MVC}
	\caption{MVC design for Rambøll Tilsyn.}
	\label{fig:MVC}
\end{figure}

Model-View-Controller er et design pattern til opbygning af user interfaces. Model er der alt
data ligger. Det er så Controllerens opgave at manipulere det til Viewet. Viewet er det som
ses af brugeren og som der kan interageres med. Hvis brugeren interagerer med applikationen er
det Controllerns opgave at få det ned i Modellen og få kaldt det nye view frem. \\
Modellen i Rambøll Tilsyn, ligger i Firebase.
Der er brugt MVC design pattern til alle views i applikationen.

\clearpage

\subsection{Navigationstræ}
Figur \ref{fig:Navi} viser et navigationstræ for Rambøll Tilsyn. Her ses flowet gennem appliaktionen og hvordan man kan til gå de forskellige views.

\begin{figure}[H] % (alternativt [H])
	\centering
	\includegraphics[height=20cm, width=12cm]{../ArkitekturDesign/Design/Navigation/Navigation}
	\caption{Navigations træ for Rambøll Tilsyn.}
	\label{fig:Navi}
\end{figure}

\clearpage
\subsection{Login} \label{sec:Login}
Dette afsnit indeholder en gennemgang af den grafiske brugergrænseflade, design og implementering af 'Login' viewet i Rambøll Tilsyn.

\subsubsection{Design}
På Figur \ref{fig:LoginSekvens} ses sekvensdiagrammet for 'Login' viewet til Rambøll Tilsyn.
\begin{figure}[H] % (alternativt [H])
	\centering
	\includegraphics[height=15cm, width=15cm]{../ArkitekturDesign/Design/Login/LoginSekvensDiagram}
	\caption{Sekvensdiagram for 'Login' i Rambøll Tilsyn.}
	\label{fig:LoginSekvens}
\end{figure}

\clearpage

\subsubsection{Grafisk brugergrænseflade}
I 'Login' viewet er der lavet felter til at bruger indtaster sit brugernavn og kodeord. Se Figur \ref{fig:LoginView}
\begin{figure}[H] % (alternativt [H])
	\centering
	\includegraphics[height=12cm, width=10cm]{../ArkitekturDesign/Design/Login/LoginView}
	\caption{'Login' viewet som det er implementeret i Rambøll Tilsyn.}
	\label{fig:LoginView}
\end{figure}

\clearpage

\subsubsection{Implementering}
I dette afsnit vil der blive beskrevet funktionaliteten for de vigtigste funktioner i koden tilhørende 'Login' viewet.

På Figur \ref{fig:Checkifusersigned} ses funktionen for CheckIfUserSignedIn().
\begin{figure}[H] % (alternativt [H])
	\centering
	\includegraphics[height=8cm, width=15cm]{../ArkitekturDesign/Design/Login/Checkifusersigned}
	\caption{Kode snip - CheckIfUserSignedIn() fra LoginViewController.cs}
	\label{fig:Checkifusersigned}
\end{figure}
Først tjekker funktionen Firebases default instance og ser om, der er en bruger, som er logget ind. \\
Er der en bruger logget ind, hoppes log ind over, og projekt listen bliver vist. \\
Ellers udskriver den at ingen bruger er logget ind.

På Figur \ref{fig:ViewDidLoadLogin} ses funktionen for ViewDidLoad().
\begin{figure}[H] % (alternativt [H])
	\centering
	\includegraphics[height=7cm, width=12cm]{../ArkitekturDesign/Design/Login/ViewDidLoad}
	\caption{Kode snip - ViewDidLoad() fra LoginViewController.cs}
	\label{fig:ViewDidLoadLogin}
\end{figure}

\clearpage

I ViewDidLoad delegeres en eventhandler\cite{Event} til LoginBtn.TouchUpInside. \\
Den indsætter placeholders i tekstfelterne e-mail og kodeord. Derudover maskerer den inputs i kodeords tekstfelt.
I ViewDidUnload fjernes eventhandleren igen, da dette er en god kode skik for at undgå memory leak\cite{Memory}.

På Figur \ref{fig:LoginBtn} ses funktionen for LoginBtn.TouchUpInside().
\begin{figure}[H] % (alternativt [H])
	\centering
	\includegraphics[height=10cm, width=18cm]{../ArkitekturDesign/Design/Login/LoginBtn}
	\caption{Kode snip - LoginBtn.TouchUpInside() fra LoginViewController.cs}
	\label{fig:LoginBtn}
\end{figure}
LoginBtn.TouchUpInside tager default instance af Firebase Auth, kalder SignIn metoden for denne instance og indsætter de to parameter skrevet i 'Login' viewet. \\
Efterfølgende valideres resultatet af log ind forsøget. Hvis der ingen fejl er, altså at e-mail og kodeord er korrekt, så vil projekt listen blive vist. \\
Der kan i switch-casen implementeres fejlhåndtering af diverse fejlkoder. 


\clearpage
Dette afsnit indeholder en gennemgang af grafisk brugergrænseflade, design og implementering af 'Project List' viewet i Rambøll Tilsyn.

\subsubsection{Design}
På figur \ref{fig:ProjctListSekvens} ses sekvens diagrammet for 'Project List' viewet til Rambøll Tilsyn.
%\begin{figure}[H] % (alternativt [H])
%	\centering
%	\includegraphics[height=15cm, width=15cm]{../ArkitekturDesign/Design/Login/LoginSekvensDiagram}
%	\caption{Sekvensdiagram for Login i Rambøll Tilsyn.}
%	\label{fig:ProjctListSekvens}
%\end{figure}

\clearpage

\subsubsection{Grafisk brugergrænseflade}
I LoginViewet er der lavet felter til at bruger indtaster sit brugernavn og kodeord. Se figur \ref{fig:ProjectListView}
\begin{figure}[H] % (alternativt [H])
	\centering
	\includegraphics[height=12cm, width=10cm]{../ArkitekturDesign/Design/Login/LoginView}
	\caption{Login viewet som det er implementeret i Rambøll Tilsyn.}
	\label{fig:ProjectListView}
\end{figure}

\clearpage

\subsubsection{Implementering}
I dette afsnit vil der blive beskrevet funktionaliteten for de vigtigste funktioner i koden tilhørende 'Login' viewet.

På figur \ref{fig:Checkifusersigned}, ses funktionen for CheckIfUserSignedIn.
\begin{figure}[H] % (alternativt [H])
	\centering
	\includegraphics[height=8cm, width=15cm]{../ArkitekturDesign/Design/Login/Checkifusersigned}
	\caption{}
	\label{fig:Checkifusersigned}
\end{figure}

\subsection{Opret Bruger}
I dette afsnit vises design, bruger grænseflade, implementering og test for 'Opret Bruger' viewet. For den fulde dokumentation henvises til Arkitektur og Design dokumentationens afsnit \ref{Design-sec:Opretbruger}.
\subsubsection{Design}
Sekvensdiagrammet for 'Opret Bruger' viewet til Rambøll Tilsyn, kan ses på figur \ref{fig:OpretBrugerSekvens}. Figuren viser det logiske flow der sker når brugeren vil oprette en bruger.
\begin{figure}[H] % (alternativt [H])
	\centering
	\includegraphics[height=18cm, width=15cm]{Design/Applikation/OpretBruger/OpretBrugerSekvensDiagram}
	\caption{Sekvensdiagram for Opret Bruger i Rambøll Tilsyn.}
	\label{fig:OpretBrugerSekvens}
\end{figure}

\subsubsection{Grafisk brugergrænseflade}
Den grafiske brugergrænseflade for 'Opret Bruger' viewet består af felter, så at bruger kan indtaste alt det information, der skal bruges når der skal oprettes en ny bruger. Se figur \ref{fig:OpretBrugerView}
\begin{figure}[H] % (alternativt [H])
	\centering
	\includegraphics[height=12cm, width=10cm]{Design/Applikation/OpretBruger/OpretBrugerView}
	\caption{Opret bruger viewet som det er implementeret i Rambøll Tilsyn.}
	\label{fig:OpretBrugerView}
\end{figure}

\subsubsection{Implementering}

\subsubsection{Test}

\clearpage
\subsection{Opret Projekt}
Dette afsnit indeholder en gennemgang af grafisk brugergrænseflade, design og implementering af Opret Projekt viewet i Rambøll Tilsyn.
\subsubsection{Grafisk brugergrænseflade}

\subsubsection{Design \& Implementering}
På figur \ref{fig:OpretProjektSekvens} ses sekvens diagrammet for opret bruger viewet til Rambøll Tilsyn.
\begin{figure}[H] % (alternativt [H])
	\centering
	\includegraphics[height=20cm, width=15cm]{../ArkitekturDesign/Design/OpretProjekt/OpretProjektSekvensDiagram}
	\caption{Sekvensdiagram for Opret Bruger i Rambøll Tilsyn.}
	\label{fig:OpretProjektSekvens}
\end{figure}
\subsection{Registrering på PDF}
Dette afsnit indeholder en gennemgang af grafisk brugergrænseflade, design og implementering af Registrering på PDF viewet i Rambøll Tilsyn.

\subsubsection{Design}
Her under kan sekvensdiagrammerne for registrering på PDF. \\
Det første sekvensdiagram, som ses på figur \ref{fig:LoadPDFSekvensDiagram}, hvordan PDF tegningen bliver loadet ind i applikationen.
\begin{figure}[H] % (alternativt [H])
	\centering
	\includegraphics[height=12cm, width=15cm]{../ArkitekturDesign/Design/RegisterPDF/LoadPDFSekvensDiagram}
	\caption{Sekvensdiagram for Registrering på PDF - Loading af PDF, i Rambøll Tilsyn.}
	\label{fig:LoadPDFSekvensDiagram}
\end{figure}

\clearpage

Næste sekvensdiagram, som ses på figur \ref{fig:LoadJSONSekvensDiagram}, viser hvordan JSON filen bliver oprettet. Sekvensen med JSON sker direkte efter at PDF sekvensen er overstået.
\begin{figure}[H] % (alternativt [H])
	\centering
	\includegraphics[height=15cm, width=15cm]{../ArkitekturDesign/Design/RegisterPDF/LoadJSONSekvensDiagram}
	\caption{Sekvensdiagram for Registrering på PDF - Loading af JSON, i Rambøll Tilsyn.}
	\label{fig:LoadJSONSekvensDiagram}
\end{figure}

\clearpage

Sidste sekvensdiagram viser hvordan systemet opføre sig, når brugeren interagere med applikationen i forbindelse med oprettelse af objekter på PDF tegningen. Denne sekvens sker i forlængelse af først Loading af PDF og Load JSON. Sekvensdiagrammet for dette kan ses på \ref{fig:RegistrerObjekterSekvensDiagram}.
\begin{figure}[H] % (alternativt [H])
	\centering
	\includegraphics[height=18cm, width=15cm]{../ArkitekturDesign/Design/RegisterPDF/RegistrerObjekterSekvensDiagram}
	\caption{Sekvensdiagram for Registrering på PDF - Registrer på PDF, i Rambøll Tilsyn.}
	\label{fig:RegistrerObjekterSekvensDiagram}
\end{figure}

\clearpage

\subsubsection{Grafisk brugergrænseflade}

\subsubsection{Implementering}
\subsection{Eksportering} \label{sec:Login}
Dette afsnit indeholder en gennemgang af grafisk brugergrænseflade, design og implementering af 'Export viewet' i Rambøll Tilsyn.

\subsubsection{Design}
På figur \ref{fig:EksporterSekvensDiagram} ses sekvens diagrammet for 'Export' viewet til Rambøll Tilsyn.
\begin{figure}[H] % (alternativt [H])
	\centering
	\includegraphics[height=10cm, width=10cm]{../ArkitekturDesign/Design/Eksportering/EksporterSekvensDiagram}
	\caption{Sekvensdiagram for Eksportering i Rambøll Tilsyn.}
	\label{fig:EksporterSekvensDiagram}
\end{figure}

\clearpage

\subsubsection{Grafisk brugergrænseflade}
Efter at have eksporteret registreringen i en excel, vil den se ud som på figur \ref{fig:Excel}.
\begin{figure}[H] % (alternativt [H])
	\centering
	\includegraphics[height=12cm, width=17cm]{../ArkitekturDesign/Design/Eksportering/Excel}
	\caption{Export viewet som det er implementeret i Rambøll Tilsyn.}
	\label{fig:Excel}
\end{figure}

\subsubsection{Implementering}

\clearpage