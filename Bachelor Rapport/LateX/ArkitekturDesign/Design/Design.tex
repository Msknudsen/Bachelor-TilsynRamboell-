\chapter{Design}
Dette kapitel indeholder en beskrivelse af design valgene for Firebase databasen og applikationen, samt hvordan disse er implementeret. \\

\section{Firebase database}

\section{Applikation}
Til udvikling af applikationen er der brugt et MVC design \cite{MVC}, som vist herunder, på figur \ref{fig:MVC}
\begin{figure}[H] % (alternativt [H])
	\centering
	\includegraphics[height=10cm, width=10cm]{../ArkitekturDesign/Design/MVC}
	\caption{MVC design for Rambøll Tilsyn.}
	\label{fig:MVC}
\end{figure}

Model-View-Controller er et design pattern til opbygning af user interfaces. Model er der alt
data ligger. Det er så Controllerens opgave at manipulere det til Viewet. Viewet er det som
ses af brugeren og som der kan interageres med. Hvis brugeren interagerer med applikationen er
det Controllerns opgave at få det ned i Modellen og få kaldt det nye view frem. \\
Der er brugt MVC design pattern til alle views i applikationen.

\clearpage

\subsection{Login} \label{sec:Login}
Dette afsnit indeholder en gennemgang af den grafiske brugergrænseflade, design og implementering af 'Login' viewet i Rambøll Tilsyn.

\subsubsection{Design}
På Figur \ref{fig:LoginSekvens} ses sekvensdiagrammet for 'Login' viewet til Rambøll Tilsyn.
\begin{figure}[H] % (alternativt [H])
	\centering
	\includegraphics[height=15cm, width=15cm]{../ArkitekturDesign/Design/Login/LoginSekvensDiagram}
	\caption{Sekvensdiagram for 'Login' i Rambøll Tilsyn.}
	\label{fig:LoginSekvens}
\end{figure}

\clearpage

\subsubsection{Grafisk brugergrænseflade}
I 'Login' viewet er der lavet felter til at bruger indtaster sit brugernavn og kodeord. Se Figur \ref{fig:LoginView}
\begin{figure}[H] % (alternativt [H])
	\centering
	\includegraphics[height=12cm, width=10cm]{../ArkitekturDesign/Design/Login/LoginView}
	\caption{'Login' viewet som det er implementeret i Rambøll Tilsyn.}
	\label{fig:LoginView}
\end{figure}

\clearpage

\subsubsection{Implementering}
I dette afsnit vil der blive beskrevet funktionaliteten for de vigtigste funktioner i koden tilhørende 'Login' viewet.

På Figur \ref{fig:Checkifusersigned} ses funktionen for CheckIfUserSignedIn().
\begin{figure}[H] % (alternativt [H])
	\centering
	\includegraphics[height=8cm, width=15cm]{../ArkitekturDesign/Design/Login/Checkifusersigned}
	\caption{Kode snip - CheckIfUserSignedIn() fra LoginViewController.cs}
	\label{fig:Checkifusersigned}
\end{figure}
Først tjekker funktionen Firebases default instance og ser om, der er en bruger, som er logget ind. \\
Er der en bruger logget ind, hoppes log ind over, og projekt listen bliver vist. \\
Ellers udskriver den at ingen bruger er logget ind.

På Figur \ref{fig:ViewDidLoadLogin} ses funktionen for ViewDidLoad().
\begin{figure}[H] % (alternativt [H])
	\centering
	\includegraphics[height=7cm, width=12cm]{../ArkitekturDesign/Design/Login/ViewDidLoad}
	\caption{Kode snip - ViewDidLoad() fra LoginViewController.cs}
	\label{fig:ViewDidLoadLogin}
\end{figure}

\clearpage

I ViewDidLoad delegeres en eventhandler\cite{Event} til LoginBtn.TouchUpInside. \\
Den indsætter placeholders i tekstfelterne e-mail og kodeord. Derudover maskerer den inputs i kodeords tekstfelt.
I ViewDidUnload fjernes eventhandleren igen, da dette er en god kode skik for at undgå memory leak\cite{Memory}.

På Figur \ref{fig:LoginBtn} ses funktionen for LoginBtn.TouchUpInside().
\begin{figure}[H] % (alternativt [H])
	\centering
	\includegraphics[height=10cm, width=18cm]{../ArkitekturDesign/Design/Login/LoginBtn}
	\caption{Kode snip - LoginBtn.TouchUpInside() fra LoginViewController.cs}
	\label{fig:LoginBtn}
\end{figure}
LoginBtn.TouchUpInside tager default instance af Firebase Auth, kalder SignIn metoden for denne instance og indsætter de to parameter skrevet i 'Login' viewet. \\
Efterfølgende valideres resultatet af log ind forsøget. Hvis der ingen fejl er, altså at e-mail og kodeord er korrekt, så vil projekt listen blive vist. \\
Der kan i switch-casen implementeres fejlhåndtering af diverse fejlkoder. 


\clearpage
\subsection{Opret Bruger}
I dette afsnit vises design, bruger grænseflade, implementering og test for 'Opret Bruger' viewet. For den fulde dokumentation henvises til Arkitektur og Design dokumentationens afsnit \ref{Design-sec:Opretbruger}.
\subsubsection{Design}
Sekvensdiagrammet for 'Opret Bruger' viewet til Rambøll Tilsyn, kan ses på figur \ref{fig:OpretBrugerSekvens}. Figuren viser det logiske flow der sker når brugeren vil oprette en bruger.
\begin{figure}[H] % (alternativt [H])
	\centering
	\includegraphics[height=18cm, width=15cm]{Design/Applikation/OpretBruger/OpretBrugerSekvensDiagram}
	\caption{Sekvensdiagram for Opret Bruger i Rambøll Tilsyn.}
	\label{fig:OpretBrugerSekvens}
\end{figure}

\subsubsection{Grafisk brugergrænseflade}
Den grafiske brugergrænseflade for 'Opret Bruger' viewet består af felter, så at bruger kan indtaste alt det information, der skal bruges når der skal oprettes en ny bruger. Se figur \ref{fig:OpretBrugerView}
\begin{figure}[H] % (alternativt [H])
	\centering
	\includegraphics[height=12cm, width=10cm]{Design/Applikation/OpretBruger/OpretBrugerView}
	\caption{Opret bruger viewet som det er implementeret i Rambøll Tilsyn.}
	\label{fig:OpretBrugerView}
\end{figure}

\subsubsection{Implementering}

\subsubsection{Test}

\clearpage
\subsection{Opret Projekt}
Dette afsnit indeholder en gennemgang af grafisk brugergrænseflade, design og implementering af Opret Projekt viewet i Rambøll Tilsyn.
\subsubsection{Grafisk brugergrænseflade}

\subsubsection{Design \& Implementering}
På figur \ref{fig:OpretProjektSekvens} ses sekvens diagrammet for opret bruger viewet til Rambøll Tilsyn.
\begin{figure}[H] % (alternativt [H])
	\centering
	\includegraphics[height=20cm, width=15cm]{../ArkitekturDesign/Design/OpretProjekt/OpretProjektSekvensDiagram}
	\caption{Sekvensdiagram for Opret Bruger i Rambøll Tilsyn.}
	\label{fig:OpretProjektSekvens}
\end{figure}
\subsection{Registrering på PDF}
Dette afsnit indeholder en gennemgang af grafisk brugergrænseflade, design og implementering af Registrering på PDF viewet i Rambøll Tilsyn.

\subsubsection{Design}
Her under kan sekvensdiagrammerne for registrering på PDF. \\
Det første sekvensdiagram, som ses på figur \ref{fig:LoadPDFSekvensDiagram}, hvordan PDF tegningen bliver loadet ind i applikationen.
\begin{figure}[H] % (alternativt [H])
	\centering
	\includegraphics[height=12cm, width=15cm]{../ArkitekturDesign/Design/RegisterPDF/LoadPDFSekvensDiagram}
	\caption{Sekvensdiagram for Registrering på PDF - Loading af PDF, i Rambøll Tilsyn.}
	\label{fig:LoadPDFSekvensDiagram}
\end{figure}

\clearpage

Næste sekvensdiagram, som ses på figur \ref{fig:LoadJSONSekvensDiagram}, viser hvordan JSON filen bliver oprettet. Sekvensen med JSON sker direkte efter at PDF sekvensen er overstået.
\begin{figure}[H] % (alternativt [H])
	\centering
	\includegraphics[height=15cm, width=15cm]{../ArkitekturDesign/Design/RegisterPDF/LoadJSONSekvensDiagram}
	\caption{Sekvensdiagram for Registrering på PDF - Loading af JSON, i Rambøll Tilsyn.}
	\label{fig:LoadJSONSekvensDiagram}
\end{figure}

\clearpage

Sidste sekvensdiagram viser hvordan systemet opføre sig, når brugeren interagere med applikationen i forbindelse med oprettelse af objekter på PDF tegningen. Denne sekvens sker i forlængelse af først Loading af PDF og Load JSON. Sekvensdiagrammet for dette kan ses på \ref{fig:RegistrerObjekterSekvensDiagram}.
\begin{figure}[H] % (alternativt [H])
	\centering
	\includegraphics[height=18cm, width=15cm]{../ArkitekturDesign/Design/RegisterPDF/RegistrerObjekterSekvensDiagram}
	\caption{Sekvensdiagram for Registrering på PDF - Registrer på PDF, i Rambøll Tilsyn.}
	\label{fig:RegistrerObjekterSekvensDiagram}
\end{figure}

\clearpage

\subsubsection{Grafisk brugergrænseflade}

\subsubsection{Implementering}