\chapter{Arktitektur}
Dette kapitel viser arkitekturen for applikationen Rambøll Tilsyn.\label{sec:Arkitektur} \\ \\ 
På Figur \ref{fig:OversigtSystembeskrivelse} ses oversigten over systemet og hvordan de forskellige elementers relationer hinanden i systemet.
\begin{figure}[H] % (alternativt [H])
	\centering
	\includegraphics[height=3cm, width=8cm]{../ArkitekturDesign/OverordnetArkitektur//Oversigtoversystem}
	\caption{Oversigt over systemet.}
	\label{fig:OversigtSystembeskrivelse}
\end{figure}
Det ses på Figur \ref{fig:OversigtSystembeskrivelse}, at brugeren benytter applikationen. Applikationen kommunikerer via internettet til databasen. \\
Yderligere beskrivelse af systemet kan findes i kravspecifikationens afsnit \ref{Krav-sec:Systembeskrivelse} Systembeskrivelse, som er vedlagt i bilag.

\clearpage

\section{Domænemodel}
Denne domænemodel giver et overblik over, hvordan systemmet skal bygges op. Domænemodellen viser forbindelserne mellem de forskellige elementer i systemmet og hvordan de interagerer med hinanden. Dette gør det lettere at gå fra domænemodellen til implementation.
I dette projekt er der udarbejdet én domænemodel som ses på Figur \ref{fig:Domain}.

\begin{figure}[H] % (alternativt [H])
	\centering
	\includegraphics[height=13cm, width=17cm]{../ArkitekturDesign/OverordnetArkitektur/Domainmodel}
	\caption{Domænemodel for Rambøll Tilsyn.}
	\label{fig:Domain}
\end{figure}
Brugeren logger ind på Rambøll Tilsyn. Her kan bruger oprette et projekt, dette projekt vil indeholde registreringer som oprettes af bruger. I denne registrering kan der tilknyttes en PDF tegning. På PDF tegningen kan brugeren oprette forskellige objekter eller fjerne forkerte objekter. \\
Når brugeren er færdig med en registrering, vil objekter, PDF-tegning mv. blive gemt i Firebase databasen.

\clearpage