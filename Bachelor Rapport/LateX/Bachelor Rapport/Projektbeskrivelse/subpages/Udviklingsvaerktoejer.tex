\section{Udviklingsværktøjer}
	Herunder vil de udviklingsværktøjer der er brugt under udviklingen af 
	produktet blive beskrevet.
	
	\subsection*{Git}
	Git\cite{GitRef} er et versionsstyrings-værktøj der i dette projekt er 
	blevet brugt til at holde styr på backup af samtlige filer. Git har under 
	udviklingen af produktet været med til at tillade flere udviklere at 
	arbejde på samme projekt/dokument uden alt for mange problemer.
	
	\subsection*{\LaTeX}
	\LaTeX \cite{LatexRef} er et opmærkningssprog brugt til tekstformatering af 
	dokumenter, hvor alt bliver skrevet i plain-tekst. Under udviklingen af 
	TrafficControl er LaTeX blevet brugt til alle dokumenter som 
	møde-referater, dokumentationen m.m. To af grundende til brug af LaTeX i 
	projektet er at det fungerer sammen med Git og at det tillader flere 
	udviklere at arbejde parallelt i samme dokument.
	
	\subsection*{Microsoft Visio}
	Microsoft Visio\cite{VisioRef} er et værktøj til udvikling af diagrammer. 
	Alle Visio dokumenter er vektorbaserede, dvs. de aldrig bliver uklare. 
	Derfor er mange diagrammer udviklet i forbindelse med Traffic Control lavet 
	i Visio.
	
	\subsection*{Draw.io}
	Draw.io\cite{Draw.io} er et online værktøj til udvikling af diagrammer. 
	Det er enkelt at anvende og mængden af figurer er omfattende, hvilket medfører at en del af diagrammerne udviklet i forbindelse med Traffic Control er tegnet i draw.io.
	
	\subsection*{Microsoft Visual Studio}
	Visual Studio\cite{VisualStudio} er et integreret udviklingsmiljø fra 
	Microsoft. Dette kan 	bruges til at udvikle alt fra konsolbaserede 
	applikationer, GUI'er i WPF til hjemmesider. Er i projektet anvendt til at 
	udvikle både Android-applikationen, webapplikationen samt alt backend.
		
	\subsection*{Postman} 
	Postman\cite{Postman} er en REST client som kører som applikation inde i 
	webbrowseren Chrome. Den er anvendt til at teste TrafficControl API'et, da 
	det er nemt at teste de forskellige REST funktionaliteter, GET, PUT, POST og 
	DELETE.
	
	\subsection*{DDS Lite}
	DDS Lite\cite{DDSLite} er et program til ER diagram modellering af relationelle databaser. 
	
	\subsection*{Certify for Windows}
	Certify\cite{Certify} er en Windows applikation, som hjælper med opsætning af Let's Encrypt for Internet Information  services. Certifikatet fornyes før udløb.
	
	\subsection*{Doxygen}
	Doxygen \cite{DoxygenInfo} Bruges til at genere dokumentation fra koden. Doxygen blev brugt i data access layer.   
	
	\subsection*{Anvendte frameworks} 
	Et framework (eller programmeringsplatform) er en betegnelse for det miljø 
	et program laves til at kunne udføres i. I udviklingen af Traffic Control 
	er flere frameworks anvendt. De forskellige frameworks beskrives i dokumentationen.
	\begin{itemize}[-]
		\item Xamarin \cite{XamarinDoc}
		\item Identity 2.0 - se afsnit \vref{Dok-sec:Identity} i dokumentationen
		\item Unity 3 - se afsnit \vref{Dok-sec:Unity3} i dokumentationen
		\item RestSharper \cite{RestSharp}
		\item NUnit \cite{NUnit}
		\item NSubstitute \cite{NSubstitute}
		\item ASP.NET MVC Framework \cite{MVC}
					
	\end{itemize}
		
	