\chapter{Erfaringer}
Dette kapitel har til formål at beskrive de samlede erfaringer for projektgruppen. 

Der er igennem projektet blevet opnået en masse erfaringer i gruppen.

Der var fra Rambølls side først et ønske om en iOS applikation, men da ingen i gruppen havde MacBooks og Rambøll ikke stillede det tilrådighed, blev det aftalt, at der skulle skrives i Xamarin, som er et cross-platform værktøj, der giver mulighed for at udvikle i iOS og Android.
I starten af projektet blev der udviklet i Xamarin Forms, da dette gjorde det muligt at genbruge både UI og Business logic mellem både iOS og Android applikationen. Da der skulle udvikles mere avanceret funktionalitet omkring PDF registrering, blev mulighederne i Xamarin Forms for begrænset\cite{Forms}. Der blev i stefet skiftet til native Xamarin.iOS udvikling, som gav mulighed for at implementere funktionaliteten.
Gruppen fandt også ud af at selvom, der udvikles i Xamarin, skulle der bruges en fysisk MacBook til at kompilere koden.

Cross-platform udviklingen har sine fordele og ulemper. At der i Xamarin f.eks. kan udvikles i C\# er en fordel, da dette er et sprog gruppen har haft undervisning i og derfor skal der ikke læres et nyt programmeringssprog. \\
En af ulemperne var, at der skulle tages højde for mange forskellige enheder i code behind for, at UI ser ens ud på alle enheder. Hvis der f.eks. blev udviklet i Android studio, kunne man få det samme layout ligegyldig, hvilken enhed man deployerede på.

Gruppen har erfaret, at iOS udvikling kan være besværlig, hvis ikke der udvikles i Swift\cite{Swift} eller Objective-C\cite{ObjC}. Hvis der skal udvikles i Swift, skal man have tilegnet sig en MacBook, som kan kører Xcode\cite{Xcode}. Det er ikke ligesom f.eks. Android Studio, som er et gratis værktøj at hente og benytte.
Hvis gruppen skulle lave dette projekt igen, ville der blive udviklet native i Swift og man vil have udviklingsværktøj i form af MacBooks tilrådighed og derved er alle problemer med byggemiljø til at kompilere koden løst.

Der kan tages mange erfaringer omkring applikationsudviklingen med til fremtidge projekter. \\