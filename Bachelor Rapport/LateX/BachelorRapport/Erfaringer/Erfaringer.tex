\chapter{Erfaringer}
Dette kapitel har til formål at beskrive de samlede erfaringer for projektgruppen. 

Igennem projektet var det planlagt at der skulle bruges Scrum, for at sikre en agil arbejdsprocess. Scrum blev efter 14 dage valgt fra, da gruppen ikke følte at Scrum var den optimale løsning.\\
Gruppen holdte stadig den agile arbejdsprocess med god kommunikation. Der blev opdelt mindre grupper med forskellige fokusområder. Disse grupper arbejdede på tværs af fokusområderne, da grupperne var afhængige af hinandens arbejde. Dette gjorde at der til tider skulle optimeres i gruppens arbejde, for at det fungerede i det samlede system.\\

Det kunne have været en fordel med en Scrum master, der opdaterede Scrum board og sprints, da disse er et godt værktøj til at skabe og holde overblik.

Til udviklingen af Android applikationen er der opnået erfaringer med brug af Xamarin, hvilket gjorde det muligt at skrive applikationen i C\#. 
\newline
Der er opnået erfaring med Client-Server arkitektur, som er brugt til udvikling af TC. 
\newline
Der har været stor fokus på sikkerhed og brugerrettigheder. Gennemskueligheden af ASP.NET MVC / Identity Framework var en stor udfordring, som tog længere tid at implementere end først forventet. Sikkerhed blev prioriteret over funktionalitet, da dette er en vigtig del for systemet, hvis det skal i drift.\\

Så alt i alt kan mange erfaringer tages med til fremtidige projekter.\\
Scrum skal der lægges noget mere fokus på, da Scrum er et godt værktøj der kan hjælpe med at opnå en fuld agil arbejdsprocess.\\
Der vil også skulle lægges mere fokus på roller, specielt i form af en team leder, som kan holde det store overblik.

I starten af projektet var der et samarbejde med Strøm Hansen. Dette forsinkede projektet, da planlægningen af første møde med Strøm Hansen trak ud, hvilket gjorde der manglede svar på opbygning og funktionaliteter af systemet. Efter mødet med Strøm Hansen, blev det besluttet at stoppe samarbejdet, da Strøm Hansen næsten havde et komplet system. Da blev det valgt at lave systemet fra bunden. 