\chapter{Erfaringer}
Dette kapitel har til formål at beskrive de samlede erfaringer for projektgruppen. 

Der er i gennem projektet blevet opnået en del erfaringer i gruppen.

Der var fra Rambølls side først ønsket en iOS applikation, men da ingen i gruppen havde MacBooks og Rambøll ikke stillede nogle tilrådighed, blev det aftalt at der skulle skrives i Xamarin, som er et cross-platform værktøj, der giver mulighed for at udvikle i iOS og Android.
I starten af projektet, blev der udviklet i Xamarin Forms, da dette gjorde det muligt, at genbruge både UI og Business logic mellem både iOS og Android applikationen. Da der skulle udvikles mere avanceret funktionalitet omkring PDF registrering, blev mulighederne i Xamarin Forms for begrænset\cite{Forms},så der blev skiftet til native Xamarin.iOS udvikling, som gav mulighed for at implementere funktionaliteten.
Gruppen fandt også ud af at selvom der udvikles i Xamarin, skulle der bruges en fysisk MacBook til at kompile koden. Dette gjorde at inden gruppen kunne komme i gang med iOS udviklingen, skulle der skaffes en MacBook. Denne skulle så opdateres til et minimums iOS.X inden at denne kunne kompile koden.

Cross-platform udvikling kan have sine fordele og ulemper. At der i Xamarin f.eks. kan udvikles i C\# er en fordel, da dette er et sprog gruppen har haft undervisning i, og man skal ikke til at lære et nyt programmeringssprog. \\
En af ulemperne kan være at man skal tage højde for mange forskellige enheder i code behind, for at UI ser ens ud på alle enheder. Hvis man f.eks. udviklede i Android studio, kunne man få det samme layout ligegyldig hvilken enhed man deployerede på.

Gruppen har erfaret at iOS udvikling kan være besværlig hvis ikke der udvikles i Swift\cite{Swift} eller Objective-C\cite{ObjC}. Hvis der skal udvikles i Swift skal man have tilegnet sig en MacBook som kan kører Xcode\cite{Xcode}. Det er ikke ligesom f.eks. Android Studio som er et gratis værktøj at hente og benytte.
Hvis gruppen skulle lave dette projekt igen, ville der blive udviklet native i Swift og man vil have udviklingsværktøj i form af MacBooks tilrådighed og derved er alle problemer med byggemiljø til at kompilere koden løst.

Der kan fra dette projekt tages mange erfaringer omkring applikations udvikling med til fremtidge projekter. \\


%I starten af projektet var der et samarbejde med Strøm Hansen. Dette forsinkede projektet, da planlægningen af første møde med Strøm Hansen trak ud, hvilket gjorde der manglede svar på opbygning og funktionaliteter af systemet. Efter mødet med Strøm Hansen, blev det besluttet at stoppe samarbejdet, da Strøm Hansen næsten havde et komplet system. Da blev det valgt at lave systemet fra bunden. 