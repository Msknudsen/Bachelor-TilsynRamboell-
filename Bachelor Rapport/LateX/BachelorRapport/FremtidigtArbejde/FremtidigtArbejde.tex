\chapter{Fremtidigt arbejde}
I dette kapitel er beskrevet hvad næste skridt ville være, hvis der skulle videreudvikles på Rambøll Tilsyn.

I forlængelse af projektet Rambøll Tilsyn, er der mange muligheder for videreudvikling og optimering.\\
Det første arbejde der skulle udføres ville være implementeringen af de user stories, som ikke er blevet implementeret i prototypen. Her tænkes bl.a. på user stories fra \textbf{Must} kategorien, da dette vil give et minimum viable product, som vil kunne gøre at Rambøll ville kunne begynde at bruge systemet, mens resten blev implementeret. \\
Når alle user stories er implementeret ville der ligge en opgave i at kunne ændre objekter, efter de er oprettet og gemt. Hvis der f.eks. oprettes et minus objekt, første gang der laves en registrering på et projekt, og dette skal ændres til et flueben, næste gang at der bliver lavet en registrering. Minus objeket skal så kunne ændres ved at bruger laver et long press på objektet. Der vil så komme et vindue op, hvor bruger kan ændre symbolet og evt skrive en kommentar. Dette vil jo så blive gemt, og vil komme med i tilsyns dokumentet når dette bliver eksporteret. \\
Der er i projektet blevet lavet et produkt med en Firebase database. Et andet punkt til fremtidig arbejde kunne være at få flyttet database delen over på Rambølls egne servere. Dette kommer selvfølgelig an på om de ønsker at have databasen intern i huset, eller om de tænker at produktet fungere bedre med Firebase databasen, så ville de blot skulle overtage administrationen af denne. 

Når systemet som beskrevet i dette projekt er færdig implementeret, ville en fremtidig feature til systemet være 3D rendering af tegninger. Dette skal kunne gøre så at brugeren i stedet for at registre på en PDF tegning, ville kunne få en 3D tegning af byggeprojektet, som der kunne laves registreringer på.

Når alt ovenstående er implementeret, minus 3D rendering, ville det sidste skridt være at overlevere systemet til Rambøll, så det kunne blive implementeret i deres systemer og arbejdsproces.

mere test og fejl håndtering build server