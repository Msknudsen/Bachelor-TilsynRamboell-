\chapter{Fremtidigt arbejde}
I dette kapitel er beskrevet hvad næste skridt ville være, hvis der skulle videreudvikles på Traffic Control (TC).

I forlængelse af projektet TC, er der mange muligheder for videreudvikling og optimering. \\
Havde der været mere tid til ændringer på prototypen, ville fokus ligge på rettelser af modellerne til Entity frameworket. De to modeller \texttt{Installation} og \texttt{Case} skal refaktoreres, for at opnå den korrekte kobling mellem disse.
Ved kommentar på sager er der kun mulighed for at tilføje en kommentar, eller ændre denne. Det skulle i stedet laves som en log, hvor hver bruger vil kunne tilføje en kommentar til loggen, som andre brugere kan tilgå. Der ønskes ydermere at kunne uploade billeder til en sag eller et lyskryds.\\
TC kan i dag kun udsende notifikationer som email, næste skridt i denne process ville være at tilføje SMS og applikations notifikationer.\\ 
En funktion som bør implementeres ville være søgning og sortering i sager samt lyskryds.

Et andet skridt i fremtidigt arbejde ville være at ændre objektet \texttt{Installation}. Her skal der være mulighed for scanning af QR koder, som sidder fysisk på lyskrydset, finde el diagrammer over lyskrydset. Der skal være en log over sager, så brugere vil kunne se historikken på lyskrydset.

Når alt dette var lavet, ville sidste skridt være at kontakte Strøm Hansen og fremvise TC, så de kan se det endelige produkt.