\chapter{Konklusion}
Dette kapitel har til formål at beskrive den samlede 
konklusion for projektet. \\

Formålet med projektet var at udvikle et system til Rambøll, for at kunne digitalisere deres registreringsproces på byggepladser rundt om i Danmark, som i dag gøres med papir og blyant. \\
Dette udmundede i produktet Rambøll Tilsyn, som består af en applikation samt en database.

Firebase står for databasedelen, samt Authentication i forhold til brugere. Dette har betydet at der kunne drages nytte af en masse funktionalitet, valdiering af brugeroplysninger, samt at holde kodeord sikre. \\
I applikationen som er implementeret som iOS applikation er der mulighed for at oprette brugere, oprette projekter, samt oprette registreringer. Dette udgøre største delen af det flow som blev designet for systemet, så en prototype ville kunne komme i pilot projekt og testes på rigtige byggepladser, mens at der blev arbejdet på et endeligt produkt. \\
Det er lykkedes at implementere de fleste essentielle features i prototypen, desværre er ikke alle features blevet implementeret. \\
Prototypen er blevet testet i et lukket miljø, og er derved ikke blevet rigtig brugertestet, udendørs på en rigtig byggeplads. Dette kan godt begrænse prototypens pålidelighed, til om den ville fungere i første test, eller der er nogle uforudsete komplikationer skulle opstå.

Rambøll Tilsyn som prototypen er i dag næsten et minimum viable product, og vil med videre arbejde
kunne sættes i drift inden for en overskuelig tidshorisont.


