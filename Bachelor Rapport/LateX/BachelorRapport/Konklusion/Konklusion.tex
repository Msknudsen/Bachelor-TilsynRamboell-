\chapter{Konklusion}
Dette kapitel har til formål, at beskrive den samlede 
konklusion for projektet. \\

Formålet med projektet var at udvikle et system til Rambøll, for at kunne digitalisere deres registreringsprocesser på byggepladser rundt om i Danmark og som i dag udføres med papir og blyant. \\
Dette udmundede i produktet Rambøll Tilsyn, som består af en applikation samt en database.

Firebase står for databasedelen og Authentication i forhold til brugere. Dette har betydet, at der kunne drages nytte af en masse funktionalitet, valdiering af brugeroplysninger samt at holde kodeord sikre. \\
I applikationen, som er implementeret som iOS applikation, er der mulighed for at oprette brugere, oprette projekter, samt oprette registreringer. Dette udgør største delen af det flow, som blev designet for systemet. En prototype vil kunne bruges som pilotprojekt og testes i rigtige arbejdssituationer, mens der blev arbejdet på et endeligt produkt. \\
Det er lykkedes at implementere de fleste essentielle funktioner i prototypen, med nogle undtagelser, herunder 'Afslut registrering på PDF' og 'Slet objekt på PDF-tegning'. \\
Prototypen er blevet testet i et lukket miljø og er derved ikke blevet brugertestet ude i rigtige arbejdssituationer. Dette kan begrænse prototypens pålidelighed til, om den vil fungere ved første test eller om skulle opstå der er nogle uforudsete komplikationer.

Rambøll Tilsyn, som prototypen er i dag næsten et minimum viable product, og vil med videre arbejde
kunne sættes i drift inden for en overskuelig tidshorisont.


