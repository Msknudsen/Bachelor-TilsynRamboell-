
	\chapter{Projektformulering}
	
	\section{Projektformulering}
	Strøm Hansen er en landsdækkende automations- og installationsleverandør, som har ansvaret for trafikstyring i Randers Kommune. \\Dette indebærer blandt andet servicering af lysregulering i vej- og gadekryds.
	Leverandøren af lyssignalerne, som er Swarco/ITS-teknik anvender en server, hvori Strøm Hansen kan hente notifikationer om evt. slukket lyskryds via mail og SMS. Derudover er der mulighed for visuelt at se status på de forskellige lyskryds via et online anlægskort.
	
	Strøm Hansen ønske, som de har fra Randers Kommune til den allerede etablererede ”log” lyder som nedenstående:
	
	\begin{itemize}[-]
		\item Randers kommune vil gerne kunne skrive bemærkninger ved de forskellige hændelser\\
		På nuværende tidspunkt har de kun læse adgang\\
		
		\item Søgning i alle hændelser ud fra dato\\
		
		\item Randers kommune ønsker en notifikation enten via. mail eller SMS når montøren opretter en ny log, for at få feedback, når opgaven er påbegyndt eller afsluttet.\\
		
		\item Anlægsdata ønskes synlig \\
		På nuværende tidspunkt er det ikke synlig for Randers kommune
	\end{itemize}
		
	Vi forestiller os at løse problemstillingen ved at tilgå data fra ITS-tekniks server. 
	Denne data gemmes i egen database hvori alle udvidelser vil blive implementeret.

	Vi vil derudover kunne udvikle en applikation til Strøm Hansens arbejdstelefoner samt PC'er.
	Denne skal notificere om fejlkoden (- eller fejlbeskrivelsen), hvorved medarbejdere har mulighed for at tjekke sig ind på pågældende opgave, hvilket andre medarbejdere kan se på applikationen. Det skal derudover være muligt for medarbejderen at ændre fejlbeskrivelse, hvis vedkommende står over for et teknisk problem, der er brug for anden assistance for at kunne løse. \\
	


