\chapter{Resultater}
Dette kapitel gennemgår resultaterne, som er opnået ved arbejde med Rambøll Tilsyn. Herunder vises en tabel over User Stories, som er opdelt efter MoSCoW, og ud fra hver User Story kan resultatet ses.
For resultaterne baseret på user story niveau henvises til Accepttestspecifikationen, som er vedlagt i bilag. \\
\begin{tabular}{ l | c  }
	\hline
	\textbf{Must} & \textbf{Resultat} \\ \hline
	Log ind (CRS-1) & Fuld funktionsdygtig \\
	\hline
	Opret bruger (CRS-2) & Fuld funktionsdygtig \\
	\hline
	Opret en registrering på PDF-tegning (CRS-4) & Fuld funktionsdygtig\\
	\hline
	Opret fluebensobjekt på PDF-tegning (CRS-5) & Fuld funktionsdygtigt\\
	\hline
	Opret minusobjekt på PDF-tegning (CRS-11) & Fuld funktionsdygtigt\\
	\hline
	Slet objekt på PDF-tegning (CRS-12) & Ikke implementeret\\
	\hline
	Afslut registrering på PDF-tegning (CRS-13) & Ikke implementeret\\
	\hline
	Opret projekt (CRS-16) & Næsten fuld funktionsdygtigt\\
	\hline
	\hline
	\textbf{Should} &  \\ \hline
	Opret billedeobjekt på PDF-tegning (CRS-6) & Ikke implementeret\\
	\hline
	Opret tekstfeltobjekt på PDF-tegning (CRS-7) & Ikke implementeret\\
	\hline
	Opret cirkelobjekt på PDF-tegning (CRS-10) & Næsten fuld funktionsdygtigt\\
	\hline
	\hline
	\textbf{Could} &  \\ \hline
	Rediger brugeroplysninger (CRS-3) & Ikke implementeret\\
	\hline
	Opret kommentarfeltobjekt på PDF-tegning (CRS-8)  & Ikke implementeret\\
	\hline
	Opret pilobjekt på PDF-tegning (CRS-9) & Ikke implementeret\\
	\hline
	\hline
	\textbf{Won't Have} & \\ \hline
	Opret en registrering uden PDF tegning (CRS-14) & Ikke implementeret\\
	\hline
	Afslut registrering uden PDF tegning (CRS-15) & Ikke implementeret\\
	\hline
	Rediger projektoplysninger (CRS-17) & Ikke implementeret\\
	\hline
	Se tilsynsrapporter (CRS-18) & Ikke implementeret\\
	\hline
	Opret sub entrerpise (CRS-19) & Ikke implementeret\\
	\hline
\end{tabular} \\


Resultatet af Rambøll Tilsyn anses som ikke helt tilfredstillende. 
Der er startet på en fin prototype til systemet. Der mangler stadig nogle enkelte funktionaliteter fra \textbf{MUST} kategorien. Hvis disse var blevet implementeret, ville prototypen have været færdig, og dette ville have givet et tilfredsstillende resultat for gruppen. \\
Der blev i slutningen af projektet lagt fokus på at få implementeret funktionalitet, så flowet igennem applikationen fungerede. Her menes der, at bruger kunne logge ind, oprette en registrering og efterfølgende lave objekter i denne registrering. \\
For at kunne gøre flowet muligt, uden at skulle hardcode brugere, valgte gruppen at få implementeret funktionalitet så der gennem applikationen kunne oprettes brugere.