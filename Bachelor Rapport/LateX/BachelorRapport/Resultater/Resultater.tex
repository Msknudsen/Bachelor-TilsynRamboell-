\chapter{Resultater}
Dette kapitel gennemgår resultaterne som er opnået ved arbejde med Rambøll Tilsyn. Herunder vises en tabel over MoSCoW analysen, med opnået resultater.\\
For resultaerne baseret på user story niveau henvises til Accepttestspecifikation, som er vedlagt i bilag. \\
\begin{tabular}{ l | c  }
	\hline
	\textbf{Must} & \textbf{Resultat} \\ \hline
	Log ind (CRS-1) & Fuld funktionsdygtig \\
	\hline
	Opret bruger (CRS-2) & Fuld funktionsdygtig \\
	\hline
	Opret en registrering på PDF tegning (CRS-4) & Fuld funktionsdygtig\\
	\hline
	Opret fluebens objekt på PDF tegning (CRS-5) & Fuld funktionsdygtigt\\
	\hline
	Opret minus objekt på PDF tegning (CRS-11) & Fuld funktionsdygtigt\\
	\hline
	Slet objekt på PDF tegning (CRS-12) & Ikke implementeret\\
	\hline
	Afslut registrering på PDF tegning (CRS-13) & Ikke implementeret\\
	\hline
	Opret projekt (CRS-16) & Næsten fuld funktionsdygtigt\\
	\hline
	\hline
	\textbf{Should} &  \\ \hline
	Opret billede objekt på PDF tegning (CRS-6) & Ikke implementeret\\
	\hline
	Opret tekstfelt objekt på PDF tegning (CRS-7) & Ikke implementeret\\
	\hline
	Opret cirkel objekt på PDF tegning (CRS-10) & Næsten fuld funktionsdygtigt\\
	\hline
	\hline
	\textbf{Could} &  \\ \hline
	Rediger af brugeroplysninger (CRS-3) & Ikke implementeret\\
	\hline
	Opret kommentarfelt objekt på PDF tegning (CRS-8)  & Ikke implementeret\\
	\hline
	Opret pil objekt på PDF tegning (CRS-9) & Ikke implementeret\\
	\hline
	\hline
	\textbf{Won't Have} & \\ \hline
	Opret en registrering uden PDF tegning (CRS-14) & Ikke implementeret\\
	\hline
	Afslut registrering uden PDF tegning (CRS-15) & Ikke implementeret\\
	\hline
	Rediger af projektoplysninger (CRS-17) & Ikke implementeret\\
	\hline
	Se tilsynsrapporter (CRS-18) & Ikke implementeret\\
	\hline
	Opret sub entrerpise (CRS-19) & Ikke implementeret\\
	\hline
\end{tabular} \\


Resultatet for Rambøll Tilsyn anses som ikke helt tilfredstillende. 
Der er startet på en fin prototype til systemet. Der mangler stadig nogle enkelte funktionaliteter fra \textbf{MUST} kategorien. Hvis disse var blevet implementeret, ville prototypen have været færdig, og dette ville have givet et tilfredsstillende resultat for gruppen. \\
Der blev i slutningen af projektet lagt fokus på at få implementeret funktionalitet, så flowet igennem applikationen fungerede. Her menes der, at bruger kunne logge ind, oprette en registrering og efterfølgende lave objekter i denne registrering. \\
For at kunne gøre flowet muligt, uden at skulle hardcode brugere, valgte gruppen at få implementeret så der gennem appliaktionen kunne oprettes brugere.