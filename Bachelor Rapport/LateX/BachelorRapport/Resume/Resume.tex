\chapter*{Resumé}
%\addcontentsline{toc}{chapter}{Resumé}
Rambøll er et landsdækkende rådgivningende ingeniør firma inden for mange forskellige områder, bl.a. byggeri, transport og miljø. \\
Deres byggetekniske afdeling i Aarhus er interesseret i at digitalisere deres registreringsproces på byggepladsen, som i dag foregår med blyant og papir.

Der er blevet anvendt en række metoder til udvikling af produktet. Disse metoder er workshops, kravspecifikationer og litteratursøgning. Der er blevet anvendt en iterativ udviklingsproces i projektet.

Afdelingen har ønsket et produkt, som gør det muligt at lade papir og blyant ligge derhjemme og i stedet benytte enten en smartphone eller tablet på byggepladser mv.

Produktet, som er blevet udviklet, er: Rambøll Tilsyn. Rambøll Tilsyn er et system bestående af en iOS applikation til smartphone eller tablet med en Firebase backend.
På applikationen vil der kunne oprettes en digitaliseret form for den analog registrering, der i dag anvendes. 

Produktet er endt i en prototype, som vil kunne videre implementeres til et endeligt produkt.

\clearpage
