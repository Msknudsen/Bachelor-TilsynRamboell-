\chapter*{Resumé}
%\addcontentsline{toc}{chapter}{Resumé}
Rambøll er et landsdækkende rådgivningende ingeniør firma inden for mange forskellige områder, bl.a. byggeri, transport og miljø. \\
Deres byggetekniske afdeling i Aarhus er interesseret i at digitalisere deres registreringsproces på byggepladsen, som i dag foregår med blyant og papir.

Der er blevet anvendt en række metoder til udviklingen af produktet. Disse metoder har hovedsagligt været workshops, kravspecifikationer og litteratursøgning. Endvidere har der været anvendt en iterativ udviklingsproces i projektet.

Afdelingen har ønsket et produkt, som gør det muligt at lade papir og blyant ligge derhjemme og i stedet benytte enten en smartphone eller tablet på byggepladser mv.

Projektet udmundede i produktet Rambøll Tilsyn, som består af en applikation samt en database.
På applikationen vil der kunne oprettes en digitaliseret form for den analog registrering. 

Produktet er endt i en prototype, som vil kunne implementeres til et endeligt produkt.

\clearpage
