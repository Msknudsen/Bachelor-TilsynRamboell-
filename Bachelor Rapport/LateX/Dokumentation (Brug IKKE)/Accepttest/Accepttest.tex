\chapter{Accepttest}\label{sec:Accepttest}
I følgende kapitel vil der gennemgåes en accepttest ud fra de skrevne user stories. I accepttesten vises der om scenarierne til de forskellige user stories er godkendt. Det er kun scenarie navnet som fremgår af accepttesten, ønskes det fulde scenarie til user storien skal man kigge i sektion \vref{sec:UserStories}

			\begin{tabular}{ l | c  }
			\textbf{User story}  & \textbf{Godkendt / Ikke godkendt} \\
			Scenarie & \\ \hline
			\textbf{Log ind} & \\
			Log ind med korrekte oplysninger - Android app  & Godkendt \\
			Log ind med korrekte oplysninger - Web app  & Godkendt \\
			Log ind med forkerte oplysninger - Android app  & Godkendt \\
			Log ind med forkerte oplysninger - Web app  & Godkendt \\ \hline
			
			\textbf{Opret bruger} & \\ 
			Opret bruger med kunde-rettigheder - Android app & Godkendt\\
			Opret bruger med montør-rettigheder - Android app & Godkendt\\
			Opret bruger med administrator-rettigheder - Android app & Godkendt\\
			Opret bruger med kunde-rettigheder - Web app & Godkendt\\
			Opret bruger med montør-rettigheder - Web app & Godkendt \\ 
			Opret bruger med administrator-rettigheder - Web app & Godkendt\\ \hline
			
			\textbf{Ændring af bruger oplysninger} & \\ 
			Ændr kode - Android app & Godkendt\\
			Ændr for- og efternavn - Android app & Godkendt\\
			Ændr telefonnummer - Android app & Godkendt\\
			Ændr notifikationsindstillinger - Android app & Godkendt\\
			Ændr kode - Web app & Godkendt \\
			Ændr for- og efternavn - Web app & Godkendt \\
			Ændr telefonnummer - Web app & Godkendt \\
			Ændr notifikationsindstillinger - Web app & Godkendt \\ \hline
			
			\textbf{Opret sag} & \\
			Opret sag - Android app & Godkendt\\
			Opret sag - Web app & Godkendt \\ \hline
			
			\textbf{Tag sag} & \\
			Tag sag - Android app & Godkendt\\
			Tag sag - Web app & Ikke godkendt \\ \hline		
		\end{tabular}
		\pagebreak
	
	\begin{tabular}{ l | c  }
		\textbf{User story}  & \textbf{Godkendt / Ikke godkendt} \\
		Scenarie & \\ \hline
		\textbf{Ændring af en driftstatus} & \\
		Ændring af en driftstatus - Android app & Godkendt\\
		Ændring af en driftstatus - Web app & Godkendt \\ \hline
		
		\textbf{Skrive en kommentar til en sag} & \\
		Skrive en kommentar til en sag - Android app & Godkendt\\
		Skrive en kommentar til en sag - Web app & Godkendt \\ \hline
		
		\textbf{Sortere sager} & \\
		Sortere sager - Android app & Ikke godkendt\\
		Sortere sager - Web app & Ikke godkendt\\ \hline	
		
		\textbf{Sortere lyskryds} & \\
		Sortere lyskryds - Android app & Ikke godkendt\\
		Sortere lyskryds - Web app & Ikke godkendt\\ \hline
		
		\textbf{Få overblik ved hjælp af map} & \\
		Få overblik ved hjælp af map - Android app & Godkendt \\
		Få overblik ved hjælp af map - Web app & Ikke godkendt \\ \hline
		
		\textbf{Modtage notifikationer om ændrede driftstatus} & \\
		Modtage notifikation på email - Android app & Godkendt\\
		Modtage notifikation på SMS & Ikke godkendt\\ \hline
		
		\textbf{Information omkring lyskryds} & \\
		Information omkring lyskryds - Android app & Godkendt\\
		Information omkring lyskryds - Web app & Ikke godkendt \\ \hline
		
		\textbf{Ikke funktionelle  krav} \\ \hline
		Skal kunne tilgås gennem en Web applikation & Godkendt\\
		og Android applikation &\\
		\hline
		Der skal anvendes Microsoft teknologier & Godkendt\\
		og software &\\
		\hline
		Alle brugere skal kunne anvende systemet & Godkendt\\
		på samme tid. Maks antal enheder er 100 & \\
		på samme tid & \\
		\hline
		Systemet skal have svartider under 500ms & Godkendt\\
		ved 99\% af requests på en dansk kablet & \\
		internetforbindelse & \\
		\hline
		Systemet skal have en oppetid på 99,7\%, & Ikke testbar\\
		målt over 3 måneder & \\
		\hline
		Database og web-server skal være køre & Godkendt\\
		på hver deres server & \\
		\hline
		Domænet trafficcontrol.dk skal anvendes & Godkendt\\
		\hline
	\end{tabular}