\chapter{Afgrænsning}
Projektet har en naturlig begrænsning i form af den korte tid fra idé til produkt, som er gældende for bachelorprojekter.\\
Smartphone applikationen vil blive udviklet ved anvendelse af Xamarin, det muliggør at skrive i C\# og er et cross platform SDK som gør det muligt at udvikle til både iOS og Android. \\
Applikationen som bliver udviklet til iOS i første omgang, da dette er Rambølls ønske. \\
Derudover er der til projektet lavet en Firebase database, for at mindske database problemer til udviklingen af appliaktionen og ikke påvirke Rambølls nuværende database. \\

\section{User stories, prioriteret efter MoSCoW} \label{sec:MoSCoW}
I denne sektion er der lavet en prioriteren af user stories efter MoSCoW analyse metoden.

\subsubsection{User stories}
\textbf{Must:} Log ind (CRS-1) \\
\textbf{Must:} Opret bruger (CRS-2) \\
\textbf{Must:} Opret en registrering på PDF tegning (CRS-4) \\
\textbf{Must:} Opret fluebens opbjekt på PDF tegning (CRS-5) \\
\textbf{Must:} Opret billede opbjekt på PDF tegning (CRS-6) \\
\textbf{Must:} Opret tekstfelt opbjekt på PDF tegning (CRS-7) \\
\textbf{Must:} Opret minus opbjekt på PDF tegning (CRS-11) \\
\textbf{Must:} Slet opbjekt på PDF tegning (CRS-12) \\
\textbf{Must:} Afslut registrering på PDF tegning (CRS-13) \\
\textbf{Must:} Opret projekt (CRS-16) \\

\textbf{Should:} Rediger af brugeroplysninger (CRS-3) \\
\textbf{Should:} Opret kommentarfelt opbjekt på PDF tegning (CRS-8) \\
\textbf{Should:} Opret pil opbjekt på PDF tegning (CRS-9) \\
\textbf{Should:} Opret cirkel opbjekt på PDF tegning (CRS-10) \\

\textbf{Could:} Opret en registrering uden PDF tegning (CRS-14) \\
\textbf{Could:} Afslut registrering uden PDF tegning (CRS-15) \\
\textbf{Could:} Rediger af projektoplysninger (CRS-17) \\
\textbf{Could:} Se tilsynsrapporter (CRS-18) \\
\textbf{Could:} Opret sub entrerpise (CRS-19)

Udfra MoSCoW analysen kan det ses hvilke user stories som der vil blive lagt fokus på først. \\
Must kategorien er den funktionalitet som skal skal implementeres under dette projekt. Should er funktionalitet som er i anden prioritet, og vil blive implementeret, hvis alle Must casene bliver færdige før afleveringsfrist. Could casene er ting som man kan arbejde videre på, hvis man ønsker at videre udvikle systemet. \\
Det forventes at alle Must casene bliver implementeret, men at Should og Could ikke bliver en del af dette projekt.