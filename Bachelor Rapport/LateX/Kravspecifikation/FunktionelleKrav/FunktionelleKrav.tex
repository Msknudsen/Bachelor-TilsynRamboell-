\chapter{Funktionelle Krav} \label{sec:FunktionelleKrav}
Følgende er en beskrivelse af de aktuelle krav stillet til projektet. De er alle opstillet som user stories. User stories er korte beskrivelser af funktionalitet, som står på en fast form. De er let læselige og har værdi for både personer direkte involveret i projektet og udefrakommende, som skal opnå en idé om kravene til projektet. 

\section{Aktør kontekst diagram}\label{sec:Aktor}
	På figur \ref{fig:AktorKontekst} ses aktør kontekst diagrammet for Rambøll Tilsyn. Diagrammet viser hvilke aktør der interagere med systemet.
\begin{figure}[H]
	\centering
	\includegraphics[width=0.6\linewidth]{FunktionelleKrav/AktorDiagram}
	\caption{Aktør kontekst diagram for Rambøll Tilsyn}
	\label{fig:AktorKontekst}
\end{figure}


\section{User stories beskrevet med Gherkin}
Til beskrivelse af user stories er Gherkin-syntaksen valgt, hvor stories 
kan skrives på et flydende sprog, mens de er testbare.\cite{Gherkin} Dette kaldes også et Business-Readable Domain  
Specific Language. En særlig fordel ved anvendelsen af syntaksen er at den er opstiller kravene så de er særligt testbare, hvilket medfører at kravene for test af features er skrevet i user stories.
Gherkin anvender nøgleord (i det flg. afsnit 
markeret med blåt), som hver især har en funktion i beskrivelsen af hver 
user story.  \vspace{0.2 cm}\\
Første del af en egenskab indeholder en beskrivelse af forretningsdomænet. 
Formatet i	dette afsnit vælges frit. Det anbefales, at man anvender et 
system, hvor der som minimum svares på hvilke aktører der har et behov, hvad 
behovet består af og hvorfor aktører har dette behov. En tre trins user 
story opfylder dette krav og opstilles med tre 
sætninger. Første sætning svarer på hvilken aktør, eller aktører der har
behovet. Dernæst svarer man på hvad det konkret er at aktøreren ønsker at 
opnå. Den tredje sætning beskriver aktørens motivation for at anvende 
funktionaliteten\\
Samtlige nøgleord beskrives nedenfor:

\large{\textbf{Givet}}\\
Disse trin definerer tilstande og datastrukturer som anvendes i de 
efterfølgende trin og scenarier.\\
\large{\textbf{Når}}\\
Disse trin beskriver handlinger som ændrer Scenariets tilstand. Dette kan 
være handlinger	foretaget af aktøreren og/eller systemet.\\
\large{\textbf{Så}}\\
Disse trin definerer udfaldet af forudgående handlinger i et givet 
scenarie. Alle scenarier skal afsluttes ved at definere et det forventede 
udfald i et eller flere "Så"trin.

\large{\textbf{Scenarier}}\\
Scenarier er samlinger af trin som definerer de funktionelle krav til 
forløb. Første scenarie i en egenskab dækker hovedfunktionaliteten. De 
efterfølgende scenarier dækker fejlhåndteringer og alternative scenarier.\\
\large{\textbf{Baggrund}}\\
Baggrund er en særlig type scenarie som udføres inden hvert af de 
efterfølgende scenarier	i egenskaben. Systemets tilstand og preconditions 
beskrives under overskriften "Baggrund" efterfulgt af et kolon og 
linjeskift. Herefter listes alle de data, tilstande og handlinger som udgør 
systemets tilstand inden samtlige af de efterfølgende Scenarier kan	udføres.


\subsection{Log ind (CRS-1)} \label{sec:USLogInd}
\textbf{\textsc{\textcolor{RoyalBlue} {Egenskab:} Log-in på applikationen}} \\
Som bruger\\
Ønsker jeg at kunne logge ind på applikationen\\
For at kunne benytte applikationen\\

\textbf{\textsc{\color{RoyalBlue}Baggrund:}}\\
\textcolor{RoyalBlue}{Givet} følgende eksisterende profil\\
\begin{tabular}{| l | l | l | l |}
	\textbf{fornavn} & \textbf{efternavn} & \textbf{e-mail} & \textbf{kode} \\
	Ao & Testbruger & Ao@mail.dk & letmein\\
\end{tabular}
\newline \newline
\textcolor{RoyalBlue}{Og} "Ao" er ikke allerede logget ind i systemet\\
\textcolor{RoyalBlue}{Og} han har navigeret til applikationens log-in side\\

\textbf{\textsc{\textcolor{RoyalBlue}{Scenarie:}} Log ind med korrekte oplysninger}\\
\textcolor{RoyalBlue}{Når} "Ao" indtaster sin mail og kode i de korrekte tekstfelter\\
\textcolor{RoyalBlue}{Og} han trykker på log-in knappen\\
\textcolor{RoyalBlue}{Så} navigeres han til applikationens forside\\

\textbf{\textsc{\textcolor{RoyalBlue}{Scenarie:}} Log ind med forkerte oplysninger} \\
\textcolor{RoyalBlue}{Når} "Ao" indtaster følgende oplysninger:\\
\begin{tabular}{| l | l |}
	\textbf{e-mail} & \textbf{kode}\\
	Ao@mail.dk & 123\\
\end{tabular}
\newline \newline

\textcolor{RoyalBlue}{Og} han trykker på log-in knappen\\
\textcolor{RoyalBlue}{Så} informeres han om, at der er indtastet forkerte oplysninger\\
\textcolor{RoyalBlue}{Og} han bliver på log-in siden\\

\subsection{Opret bruger (CRS-2)} \label{sec:USOpretBruger}
\textbf{\textsc{\textcolor{RoyalBlue}{Egenskab:} Opret bruger}} \\
Som bruger\\
Ønsker jeg at kunne oprette en bruger på applikationen\\
For at kunne give en anden bruger adgang til systemet \\

\textcolor{RoyalBlue}{\textbf{\textsc{Baggrund:}}}\\
\textcolor{RoyalBlue}{Givet} at brugeren "Rasmus" er logget ind \\
\textcolor{RoyalBlue}{Og} han vil oprette en bruger med følgende oplysninger:\\
\begin{tabular}{| l | l |}
	\textbf{e-mail} & \textbf{kode} \\
	jakob@ramboell.dk & solskin\\
\end{tabular}
\newline \newline
\textcolor{RoyalBlue}{Og} han er navigeret til opret bruger siden for at oprette brugere  \\

\textbf{\textsc{\textcolor{RoyalBlue}{Scenarie:}} Opret bruger}\\
\textcolor{RoyalBlue}{Når} "Rasmus" indtaster mail, kode, telefonnummer, 
fornavn og efternavn \\
\textcolor{RoyalBlue}{Og} trykker på opret bruger-knappen\\
\textcolor{RoyalBlue}{Så} oprettes den nye bruger\\
\textcolor{RoyalBlue}{Og} den nye bruger kan logge ind\\

\subsection{Rediger af brugeroplysninger (CRS-3)} \label{sec:USRedigerBruger}
\textbf{\textsc{\textcolor{RoyalBlue}{Egenskab:} Rediger af brugeroplysninger}}\\
Som bruger\\
Ønsker jeg at kunne ændre brugeroplysninger\\
For at have aktuelle oplysninger

\textsc{\textcolor{RoyalBlue}{\textbf{Baggrund:}}}\\
\textcolor{RoyalBlue}{Givet} at "Niels" er logget ind\\
\textcolor{RoyalBlue}{Og} han har følgende nuværende oplysninger:\\
\begin{tabular}{| l | l | l |}
	\textbf{fornavn} & \textbf{efternavn} & \textbf{kodeord} \\
	Niels & Testbruger & paris \\
\end{tabular}
\newline \newline
\textcolor{RoyalBlue}{Og} at han vil ændre indstillingerne til følgende:\\
\begin{tabular}{| l | l | l | l |}
	\textbf{fornavn} & \textbf{efternavn} & \textbf{kodeord} \\
	Niels & TestTest & london \\
\end{tabular}
\newline \newline
\textcolor{RoyalBlue}{Og} at han er navigeret til rediger bruger siden \\

\textbf{\textsc{\textcolor{RoyalBlue}{Scenarie:}} Rediger kode}\\-
\textcolor{RoyalBlue}{Når} Niels indtaster "paris" i kodefeltet "Gammel adgangskode"\\
\textcolor{RoyalBlue}{Og} han skriver "london" i "Ny adgangskode" og "Bekræft ny adgangskode"\\
\textcolor{RoyalBlue}{Og} han trykker på gem oplysninger\\
\textcolor{RoyalBlue}{Så} gemmes oplysningerne\\
\textcolor{RoyalBlue}{Og} han informeres om at ændringerne er gemt

\textbf{\textsc{\textcolor{RoyalBlue}{Scenarie:}} Rediger for- og efternavn}\\
\textcolor{RoyalBlue}{Når} Niels indtaster nyt for- og efternavn i de korrekte tekstfelter\\
\textcolor{RoyalBlue}{Og} han trykker på ok\\
\textcolor{RoyalBlue}{Så} gemmes oplysningerne\\
\textcolor{RoyalBlue}{Og} han kan se at ændringerne er gemt \\

\textbf{\textsc{\textcolor{RoyalBlue}{Scenarie:}} Rediger telefonnummer}\\
\textcolor{RoyalBlue}{Når} Niels indtaster "87654321" i det korrekte tekstfelt\\
\textcolor{RoyalBlue}{Og} han trykker på ok\\
\textcolor{RoyalBlue}{Så} gemmes oplysningerne\\
\textcolor{RoyalBlue}{Og} han kan se at ændringerne er gemt \\

\subsection{Opret en registrering på PDF tegning (CRS-4)} \label{sec:USOpretRegPåPDF}
\textbf{\textsc{\textcolor{RoyalBlue}{Egenskab:} Opret en registrering på PDF tegning}}\\
Som bruger\\
Ønsker jeg at kunne oprette en registrering på en PDF\\
For kunne lave en registrering på et givent projekts pdf tegning \\

\textsc{\textcolor{RoyalBlue}{\textbf{Baggrund:}}}\\
\textcolor{RoyalBlue}{Givet} at "Søren" er logget ind\\
\textcolor{RoyalBlue}{Og} han har følgende nuværende oplysninger:\\
\begin{tabular}{| l | l |}
	\textbf{fornavn} & \textbf{rolle} \\
	Søren & Bruger\\
\end{tabular}

\textbf{\textsc{\textcolor{RoyalBlue}{Scenarie:}} Opret en registrering på PDF tegning}\\
\textcolor{RoyalBlue}{Når} Søren vælger "Opret registrering på PDF tegning"\\
\textcolor{RoyalBlue}{Så}  navigeres han videre til registrerings siden\\
\textcolor{RoyalBlue}{Så}  har han mulighed for at indsætte objekter på PDF'en som en del af hans registrering\\
\textcolor{RoyalBlue}{Når} han er færdig med at registrer kan han vælge afslut \\
\textcolor{RoyalBlue}{Så}  navigeres hans videre til fejl rapporten for alle objekter oprettet på projektet \\

\subsection{Opret fluebens opbjekt på PDF tegning (CRS-5)} \label{sec:USOpretFlueben}
\textbf{\textsc{\textcolor{RoyalBlue}{Egenskab:} Opret fluebens opbjekt på PDF tegning}}\\
Som bruger\\
Ønsker jeg at kunne oprette et fluebens objekt på en PDF\\
For kunne redigere min registrering på et givent projekts pdf tegning \\

\textsc{\textcolor{RoyalBlue}{\textbf{Baggrund:}}}\\
\textcolor{RoyalBlue}{Givet} at "Søren" er logget ind\\
\textcolor{RoyalBlue}{Og} han har følgende nuværende oplysninger:\\
\begin{tabular}{| l | l |}
	\textbf{fornavn} & \textbf{rolle} \\
	Søren & Bruger\\
\end{tabular}

\textbf{\textsc{\textcolor{RoyalBlue}{Scenarie:}} Opret fluebens opbjekt på PDF tegning}\\
\textcolor{RoyalBlue}{Når} Søren trykker på fluebens objektet\\
\textcolor{RoyalBlue}{Så}  får han valgmuligheden om hvilken farve han ønsker fluebenet skal have\\
\textcolor{RoyalBlue}{Så}  har han mulighed for sætte objektet det sted på PDF tegningen han ønsker\\
\textcolor{RoyalBlue}{Når} han har placeret flueben objektet, kan han oprette et nyt objekt, slette et objekt eller vælge at afslutte \\

\subsection{Opret billede opbjekt på PDF tegning (CRS-6)} \label{sec:USOpretBillede}
\textbf{\textsc{\textcolor{RoyalBlue}{Egenskab:} Opret billede opbjekt på PDF tegning}}\\
Som bruger\\
Ønsker jeg at kunne oprette et billede objekt på en PDF\\
For kunne redigere min registrering på et givent projekts pdf tegning \\

\textsc{\textcolor{RoyalBlue}{\textbf{Baggrund:}}}\\
\textcolor{RoyalBlue}{Givet} at "Jan" er logget ind\\
\textcolor{RoyalBlue}{Og} han har følgende nuværende oplysninger:\\
\begin{tabular}{| l | l |}
	\textbf{fornavn} & \textbf{rolle} \\
	Jan & Bruger\\
\end{tabular}

\textbf{\textsc{\textcolor{RoyalBlue}{Scenarie:}} Opret fluebens opbjekt på PDF tegning}\\
\textcolor{RoyalBlue}{Når} Jan trykker på billede objektet\\
\textcolor{RoyalBlue}{Så}  har han mulighed for sætte objektet det sted på PDF tegningen han ønsker\\
\textcolor{RoyalBlue}{Så}  åbnes hans kamera og han kan nu tilføje et billede, og derefter en billede tekst\\
\textcolor{RoyalBlue}{Når} han har placeret billede objektet, kan han oprette et nyt objekt, slette et objekt eller vælge at afslutte \\

\subsection{Opret tekstfelt opbjekt på PDF tegning (CRS-7)} \label{sec:USOpretTekstfelt}
\textbf{\textsc{\textcolor{RoyalBlue}{Egenskab:} Opret tekstfelt opbjekt på PDF tegning}}\\
Som bruger\\
Ønsker jeg at kunne oprette et tekstfelt objekt på en PDF\\
For kunne redigere min registrering på et givent projekts pdf tegning\\

\textsc{\textcolor{RoyalBlue}{\textbf{Baggrund:}}}\\
\textcolor{RoyalBlue}{Givet} at "Jan" er logget ind\\
\textcolor{RoyalBlue}{Og} han har følgende nuværende oplysninger:\\
\begin{tabular}{| l | l |}
	\textbf{fornavn} & \textbf{rolle} \\
	Jan & Bruger\\
\end{tabular}

\textbf{\textsc{\textcolor{RoyalBlue}{Scenarie:}} Opret tekstfelt opbjekt på PDF tegning}\\
\textcolor{RoyalBlue}{Når} Jan trykker på tekstfelt objektet\\
\textcolor{RoyalBlue}{Så}  har han mulighed for sætte objektet det sted på PDF tegningen han ønsker\\
\textcolor{RoyalBlue}{Så}  har han mulighed for at skrive en tekst, osm ligger som et objekt han kan åbne og læse teksten\\
\textcolor{RoyalBlue}{Når} han har placeret teskstfelt objektet, kan han oprette et nyt objekt, slette et objekt eller vælge at afslutte \\

\subsection{Opret kommentarfelt opbjekt på PDF tegning (CRS-8)} \label{sec:USOpretKommentarfelt}
\textbf{\textsc{\textcolor{RoyalBlue}{Egenskab:} Opret kommentarfelt opbjekt på PDF tegning}}\\
Som bruger\\
Ønsker jeg at kunne oprette et kommentarfelt objekt på en PDF\\
For kunne redigere min registrering på et givent projekts pdf tegning\\

\textsc{\textcolor{RoyalBlue}{\textbf{Baggrund:}}}\\
\textcolor{RoyalBlue}{Givet} at "Morten" er logget ind\\
\textcolor{RoyalBlue}{Og} han har følgende nuværende oplysninger:\\
\begin{tabular}{| l | l |}
	\textbf{fornavn} & \textbf{rolle} \\
	Morten & Bruger\\
\end{tabular}

\textbf{\textsc{\textcolor{RoyalBlue}{Scenarie:}} Opret kommentarfelt opbjekt på PDF tegning}\\
\textcolor{RoyalBlue}{Når} Morten trykker på kommentarfelt objektet\\
\textcolor{RoyalBlue}{Så}  har han mulighed for sætte objektet det sted på PDF tegningen han ønsker\\
\textcolor{RoyalBlue}{Så}  har han mulighed for at skrive en tekst, som vises på PDF tegningen\\
\textcolor{RoyalBlue}{Når} han har placeret teskstfelt objektet, kan han oprette et nyt objekt, slette et objekt eller vælge at afslutte \\

\subsection{Opret pil opbjekt på PDF tegning (CRS-9)} \label{sec:USOpretPilObjekt}
\textbf{\textsc{\textcolor{RoyalBlue}{Egenskab:} Opret pil opbjekt på PDF tegning}}\\
Som bruger\\
Ønsker jeg at kunne oprette et pil objekt på en PDF\\
For kunne redigere min registrering på et givent projekts pdf tegning\\

\textsc{\textcolor{RoyalBlue}{\textbf{Baggrund:}}}\\
\textcolor{RoyalBlue}{Givet} at "Morten" er logget ind\\
\textcolor{RoyalBlue}{Og} han har følgende nuværende oplysninger:\\
\begin{tabular}{| l | l |}
	\textbf{fornavn} & \textbf{rolle} \\
	Morten & Bruger\\
\end{tabular}

\textbf{\textsc{\textcolor{RoyalBlue}{Scenarie:}} Opret pil opbjekt på PDF tegning}\\
\textcolor{RoyalBlue}{Når} Morten trykker på pil objektet\\
\textcolor{RoyalBlue}{Så}  har han mulighed for sætte objektet det sted på PDF tegningen han ønsker\\
\textcolor{RoyalBlue}{Så}  har han mulighed for at trække i pilen så den passer i længde og retning\\
\textcolor{RoyalBlue}{Når} han har placeret pil objektet, kan han oprette et nyt objekt, slette et objekt eller vælge at afslutte \\

\subsection{Opret cirkel opbjekt på PDF tegning (CRS-10)} \label{sec:USOpretCirkelObjekt}
\textbf{\textsc{\textcolor{RoyalBlue}{Egenskab:} Opret cirkel opbjekt på PDF tegning}}\\
Som bruger\\
Ønsker jeg at kunne oprette et cirkel objekt på en PDF\\
For kunne redigere min registrering på et givent projekts pdf tegning\\

\textsc{\textcolor{RoyalBlue}{\textbf{Baggrund:}}}\\
\textcolor{RoyalBlue}{Givet} at "Morten" er logget ind\\
\textcolor{RoyalBlue}{Og} han har følgende nuværende oplysninger:\\
\begin{tabular}{| l | l |}
	\textbf{fornavn} & \textbf{rolle} \\
	Morten & Bruger\\
\end{tabular}

\textbf{\textsc{\textcolor{RoyalBlue}{Scenarie:}} Opret cirkel opbjekt på PDF tegning}\\
\textcolor{RoyalBlue}{Når} Morten trykker på cirkel objektet\\
\textcolor{RoyalBlue}{Så}  har han mulighed for sætte objektet det sted på PDF tegningen han ønsker\\
\textcolor{RoyalBlue}{Så}  har han mulighed for at trække i cirkelen så den passer i radius\\
\textcolor{RoyalBlue}{Når} han har placeret cirkel objektet, kan han oprette et nyt objekt, slette et objekt eller vælge at afslutte \\

\subsection{Opret minus opbjekt på PDF tegning (CRS-11)} \label{sec:USOpretMinusObjekt}
\textbf{\textsc{\textcolor{RoyalBlue}{Egenskab:} Opret minus opbjekt på PDF tegning}}\\
Som bruger\\
Ønsker jeg at kunne oprette et minus objekt på en PDF\\
For kunne redigere min registrering på et givent projekts pdf tegning\\

\textsc{\textcolor{RoyalBlue}{\textbf{Baggrund:}}}\\
\textcolor{RoyalBlue}{Givet} at "Lars" er logget ind\\
\textcolor{RoyalBlue}{Og} han har følgende nuværende oplysninger:\\
\begin{tabular}{| l | l |}
	\textbf{fornavn} & \textbf{rolle} \\
	Lars & Bruger\\
\end{tabular}

\textbf{\textsc{\textcolor{RoyalBlue}{Scenarie:}} Opret minus opbjekt på PDF tegning}\\
\textcolor{RoyalBlue}{Når} Lars trykker på minus objektet\\
\textcolor{RoyalBlue}{Så}  har han mulighed for sætte objektet det sted på PDF tegningen han ønsker\\
\textcolor{RoyalBlue}{Så}  har han mulighed for at placere den tilhørende cirkel, hvis han ønsker at markere et større område\\
\textcolor{RoyalBlue}{Når} han har placeret minus objektet, kan han oprette et nyt objekt, slette et objekt eller vælge at afslutte \\

\subsection{Slet opbjekt på PDF tegning (CRS-12)} \label{sec:USSletObjekt}
\textbf{\textsc{\textcolor{RoyalBlue}{Egenskab:} Slet opbjekt på PDF tegning}}\\
Som bruger\\
Ønsker jeg at kunne slette et objekt på en PDF\\
For kunne redigere min registrering på et givent projekts pdf tegning\\

\textsc{\textcolor{RoyalBlue}{\textbf{Baggrund:}}}\\
\textcolor{RoyalBlue}{Givet} at "Lars" er logget ind\\
\textcolor{RoyalBlue}{Og} han har følgende nuværende oplysninger:\\
\begin{tabular}{| l | l |}
	\textbf{fornavn} & \textbf{rolle} \\
	Lars & Bruger\\
\end{tabular}

\textbf{\textsc{\textcolor{RoyalBlue}{Scenarie:}} Slet opbjekt på PDF tegning}\\
\textcolor{RoyalBlue}{Når} Lars trykker på viskelæder objektet\\
\textcolor{RoyalBlue}{Så}  har han mulighed for slette et objektet på PDF tegningen\\
\textcolor{RoyalBlue}{Så}  har han mulighed for at trykke på det objekt han ønsker at slette\\
\textcolor{RoyalBlue}{Når} han har slettet objektet, kan han oprette et nyt objekt, slette et objekt eller vælge at afslutte \\

\subsection{Afslut registrering på PDF tegning (CRS-13)} \label{sec:USAfslutRegPåPDF}
\textbf{\textsc{\textcolor{RoyalBlue}{Egenskab:} Afslut registrering på PDF tegning}}\\
Som bruger\\
Ønsker jeg at kunne slette et objekt på en PDF\\
For kunne redigere min registrering på et givent projekts pdf tegning\\

\textsc{\textcolor{RoyalBlue}{\textbf{Baggrund:}}}\\
\textcolor{RoyalBlue}{Givet} at "Jakob" er logget ind\\
\textcolor{RoyalBlue}{Og} han har følgende nuværende oplysninger:\\
\begin{tabular}{| l | l |}
	\textbf{fornavn} & \textbf{rolle} \\
	Jakob & Bruger\\
\end{tabular}
\newline

\textbf{\textsc{\textcolor{RoyalBlue}{Scenarie:}} Afslut registrering på PDF tegning}\\
\textcolor{RoyalBlue}{Når} Jakob trykker på submenu ikonet\\
\textcolor{RoyalBlue}{Så}  har han mulighed for afslut og eksporter\\
\textcolor{RoyalBlue}{Så}  navigeres han tilbage, og hans mail skulle gerne åbne med en vedhæftet excel fil\\

\subsection{Opret en registrering uden PDF tegning (CRS-14)} \label{sec:USOpretRegUdenPDF}
\textbf{\textsc{\textcolor{RoyalBlue}{Egenskab:} Opret en registrering uden PDF tegning}}\\
Som bruger\\
Ønsker jeg at kunne oprette en registrering uden en PDF\\
For kunne lave en registrering på et givent projekt\\

\textsc{\textcolor{RoyalBlue}{\textbf{Baggrund:}}}\\
\textcolor{RoyalBlue}{Givet} at "Dan" er logget ind\\
\textcolor{RoyalBlue}{Og} han har følgende nuværende oplysninger:\\
\begin{tabular}{| l | l |}
	\textbf{fornavn} & \textbf{rolle} \\
	Dan & Bruger\\
\end{tabular}
\newline

\textbf{\textsc{\textcolor{RoyalBlue}{Scenarie:}} Opret en registrering uden PDF tegning}\\
\textcolor{RoyalBlue}{Når} Dan vælger "Opret registrering uden PDF tegning"\\
\textcolor{RoyalBlue}{Så} navigeres han videre til registrerings siden\\
\textcolor{RoyalBlue}{Så} har han mulighed for at indsætte billede objekter på den tomme PDF\\

\subsection{Afslut registrering uden PDF tegning (CRS-15)} \label{sec:USAfslutRegUdenPDF}
\textbf{\textsc{\textcolor{RoyalBlue}{Egenskab:} Afslut registrering på PDF tegning}}\\
Som bruger\\
Ønsker jeg at kunne slette et objekt på en PDF\\
For kunne redigere min registrering på et givent projekts pdf tegning\\

\textsc{\textcolor{RoyalBlue}{\textbf{Baggrund:}}}\\
\textcolor{RoyalBlue}{Givet} at "Jakob" er logget ind\\
\textcolor{RoyalBlue}{Og} han har følgende nuværende oplysninger:\\
\begin{tabular}{| l | l |}
	\textbf{fornavn} & \textbf{rolle} \\
	Jakob & Bruger\\
\end{tabular}

\textbf{\textsc{\textcolor{RoyalBlue}{Scenarie:}} Afslut registrering på PDF tegning}\\
\textcolor{RoyalBlue}{Når} Jakob trykker på submenu ikonet\\
\textcolor{RoyalBlue}{Så}  har han mulighed for afslut og eksporter\\
\textcolor{RoyalBlue}{Så}  navigeres han tilbage, og hans mail skulle gerne åbne med en vedhæftet word fil\\


\textbf{\textsc{\textcolor{RoyalBlue}{Scenarie:}} Se tilsynsrapporter}\\
\textcolor{RoyalBlue}{Når} Morten vælger et givent projekt\\
\textcolor{RoyalBlue}{Så} kan han nu vælge at se en liste over alle oprettede fejlrapporter\\
\textcolor{RoyalBlue}{Og} derved se alle fejl oprettet på projektet\\


\subsection{Opret projekt (CRS-16)} \label{sec:USOpretProjekt}
\textbf{\textsc{\textcolor{RoyalBlue}{Egenskab:} Opret projekt}}\\
Som bruger\\
Ønsker jeg at kunne oprette projektoplysninger\\
For at have aktuelle oplysninger

\textsc{\textcolor{RoyalBlue}{\textbf{Baggrund:}}}\\
\textcolor{RoyalBlue}{Givet} at "Jonas" er logget ind\\
\textcolor{RoyalBlue}{Og} han ønsker at oprette et projekt med følgende oplysninger:\\
\begin{tabular}{| l | l | l |}
	\textbf{Projektnavn} & \textbf{Projektnummer} & \textbf{Adresse} \\
	Bro & Projekt 170 & Finlandsgade 22 \\
\end{tabular}
\newline \newline

\textbf{\textsc{\textcolor{RoyalBlue}{Scenarie:}} Opret projekt}\\
\textcolor{RoyalBlue}{Når} Alexander indtaster projekt navn i projekt navn feltet\\
\textcolor{RoyalBlue}{Og} han forlader tekstfeltet\\
\textcolor{RoyalBlue}{Så} gemmes oplysningerne\\
\textcolor{RoyalBlue}{Når} Alexander indtaster projekt nummer i projekt nummer feltet\\
\textcolor{RoyalBlue}{Og} han forlader tekstfeltet\\
\textcolor{RoyalBlue}{Så} gemmes oplysningerne\\
\textcolor{RoyalBlue}{Når} Alexander indtaster projekt adresse i projekt adresse feltet\\
\textcolor{RoyalBlue}{Og} han forlader tekstfeltet\\
\textcolor{RoyalBlue}{Så} gemmes oplysningerne\\
\textcolor{RoyalBlue}{Når} Alexander indtaster projektets entreprenør informationer i projekt entreprenør felter\\
\textcolor{RoyalBlue}{Og} han forlader tekstfeltet\\
\textcolor{RoyalBlue}{Så} gemmes oplysningerne\\

\subsection{Rediger af projektoplysninger (CRS-17)} \label{sec:USRedigerProjekt}
\textbf{\textsc{\textcolor{RoyalBlue}{Egenskab:} Rediger af projektoplysninger}}\\
Som bruger\\
Ønsker jeg at kunne ændre projektoplysninger\\
For at have aktuelle oplysninger

\textsc{\textcolor{RoyalBlue}{\textbf{Baggrund:}}}\\
\textcolor{RoyalBlue}{Givet} at "Alexander" er logget ind\\
\textcolor{RoyalBlue}{Og} han har et projekt nuværende oplysninger:\\
\begin{tabular}{| l | l | l |}
	\textbf{Projektnavn} & \textbf{Projektnummer} & \textbf{Adresse} \\
	Bro & Projekt 170 & Finlandsgade 22 \\
\end{tabular}
\newline \newline
\textcolor{RoyalBlue}{Og} at han vil ændre indstillingerne til følgende:\\
\begin{tabular}{| l | l | l | l |}
	\textbf{Projektnavn} & \textbf{Projektnummer} & \textbf{Adresse} \\
	Bro - Olaf Palme & Projekt 170 & Olaf Palmes Alle 22 \\
\end{tabular}
\newline

\textbf{\textsc{\textcolor{RoyalBlue}{Scenarie:}} Rediger projekt navn}\\
\textcolor{RoyalBlue}{Når} Alexander indtaster det nye projekt navn i projekt navn feltet\\
\textcolor{RoyalBlue}{Og} han forlader tekstfeltet\\
\textcolor{RoyalBlue}{Så} gemmes oplysningerne\\

\textbf{\textsc{\textcolor{RoyalBlue}{Scenarie:}} Rediger projekt nummer}\\
\textcolor{RoyalBlue}{Når} Alexander indtaster det nye projekt nummer i projekt nummer feltet\\
\textcolor{RoyalBlue}{Og} han forlader tekstfeltet\\
\textcolor{RoyalBlue}{Så} gemmes oplysningerne\\

\textbf{\textsc{\textcolor{RoyalBlue}{Scenarie:}} Rediger projekt adresse}\\
\textcolor{RoyalBlue}{Når} Alexander indtaster det nye projekt adresse i projekt adresse feltet\\
\textcolor{RoyalBlue}{Og} han forlader tekstfeltet\\
\textcolor{RoyalBlue}{Så} gemmes oplysningerne\\

\textbf{\textsc{\textcolor{RoyalBlue}{Scenarie:}} Rediger entreprenør information}\\
\textcolor{RoyalBlue}{Når} Alexander indtaster det nye navn på entreprenøren i tekstfeltet for entreprenør navn \\
\textcolor{RoyalBlue}{Og} han forlader tekstfeltet\\
\textcolor{RoyalBlue}{Så} gemmes oplysningerne\\
\textcolor{RoyalBlue}{Når} Alexander indtaster den nye email på entreprenøren i tekstfeltet for entreprenør mail \\
\textcolor{RoyalBlue}{Og} han forlader tekstfeltet\\
\textcolor{RoyalBlue}{Så} gemmes oplysningerne\\
\textcolor{RoyalBlue}{Når} Alexander indtaster det nye telefonnummer på entreprenøren i tekstfeltet for entreprenør telefon \\
\textcolor{RoyalBlue}{Og} han forlader tekstfeltet\\
\textcolor{RoyalBlue}{Så} gemmes oplysningerne\\

\subsection{Se tilsynsrapporter (CRS-18)} \label{sec:USTilsynsrapport}
\textbf{\textsc{\textcolor{RoyalBlue}{Egenskab:} Se tilsynsrapporter}}\\
Som bruger\\
Ønsker jeg at kunne se tilsynsrapporter for et givent projekt\\
For at kunne følge byggeriets gang

\textsc{\textcolor{RoyalBlue}{\textbf{Baggrund:}}}\\
\textcolor{RoyalBlue}{Givet} at "Jonas" er logget ind\\
\textcolor{RoyalBlue}{Og} han har følgende nuværende oplysninger:\\
\begin{tabular}{| l | l |}
	\textbf{fornavn} & \textbf{rolle} \\
	Jonas & Bruger\\
\end{tabular}
\newline

\textbf{\textsc{\textcolor{RoyalBlue}{Scenarie:}} Se tilsynsrapporter}\\
\textcolor{RoyalBlue}{Når} Jonas vælger tilsynsrapport\\
\textcolor{RoyalBlue}{Så} navigeres han ind til en liste af alle oprettede tilsynsrapporter\\
\textcolor{RoyalBlue}{Så} kan han vælge hvilken han ønsker at exportere til excel\\
\textcolor{RoyalBlue}{Når} han har valgt hvilken han ønsker at exportere til excel, vil han blive navigeret tilbage til projektforsiden\\

\subsection{Opret sub entrerpise (CRS-19)} \label{sec:USOpretSubEntreprise}
\textbf{\textsc{\textcolor{RoyalBlue}{Egenskab:} Opret sub entrerpise}}\\
Som bruger\\
Ønsker jeg at kunne oprette projektoplysninger\\
For at have aktuelle oplysninger \\

\textsc{\textcolor{RoyalBlue}{\textbf{Baggrund:}}}\\
\textcolor{RoyalBlue}{Givet} at "Jonas" er logget ind\\
\textcolor{RoyalBlue}{Og} han ønsker at oprette en sub entreprise på et projekt med følgende oplysninger:\\
\begin{tabular}{| l | l | l |}
	\textbf{Projektnavn} & \textbf{Projektnummer} & \textbf{Adresse} \\
	Broer & Projekt 170 & Finlandsgade 22 \\
\end{tabular}
\newline \newline

\begin{tabular}{| l | l | l |}
	\textbf{Entreprise} & \textbf{Entreprisetnummer} & \textbf{Adresse} \\
	Bro 1 & Projekt 170.1 & Finlandsgade 28 \\
\end{tabular}
\newline \newline

\textbf{\textsc{\textcolor{RoyalBlue}{Scenarie:}} Opret sub entrerpise}\\
\textcolor{RoyalBlue}{Når} Alexander indtaster entreprise navn i entreprise navn feltet\\
\textcolor{RoyalBlue}{Og} han forlader tekstfeltet\\
\textcolor{RoyalBlue}{Så} gemmes oplysningerne\\
\textcolor{RoyalBlue}{Når} Alexander indtaster entreprise nummer i entreprise nummer feltet\\
\textcolor{RoyalBlue}{Og} han forlader tekstfeltet\\
\textcolor{RoyalBlue}{Så} gemmes oplysningerne\\
\textcolor{RoyalBlue}{Når} Alexander indtaster entreprise adresse i entreprise adresse feltet\\
\textcolor{RoyalBlue}{Og} han forlader tekstfeltet\\
\textcolor{RoyalBlue}{Så} gemmes oplysningerne\\
\textcolor{RoyalBlue}{Når} Alexander indtaster entreprise informationer i entreprise informations felter\\
\textcolor{RoyalBlue}{Og} han forlader tekstfeltet\\
\textcolor{RoyalBlue}{Så} gemmes oplysningerne\\
