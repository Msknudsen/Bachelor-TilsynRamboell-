\chapter{Projektformulering}

Rambøll er Danmarks største rådgivende ingeniør virksomhed og beskæftiger på verdensplan over 13.000
medarbejdere. Kontoret i Aarhus er Danmarks næststørste med mere end 500 ansatte.\\
Afdelingen, som du/I vil være i kontakt med, består af 16 medarbejdere, hvis primære arbejdsopgaver ligger i
den udførende del af bygge- og anlægsopgaverne. Af nuværende opgaver kan nævnes: Helhedsplan GellerupToveshøj,
Aarhus Letbane og Det ny Skejby sygehus.\\
Meget af arbejdet foregår på byggepladser med at registrere alt fra den tekniske udførelse til arbejdsmiljøet for
håndværkerne.\\
I dag foregår dette arbejde for langt de flestes vedkommende med papir og blyant suppleret med et
telefonkamera.
Det er denne proces vi ønsker at digitalisere via en app til smartphone eller tablet.
Der findes flere apps i Appstore, der forsøger at løfte opgaven, men der er ingen af dem, der til fulde kan det, vi
ønsker. Eksempler på apps, der har de basale funktionaliteter, men lige mangler det sidste, er:
\begin{itemize}[-]
	\item iGIS for iPad 
	\item BIMx - Building Information Model eXplorer
	\item Avenza Maps
\end{itemize}

Vi er meget åbne for, hvordan app’en kan udformes og vi forestiller os, at vi i sammenarbejde med jer indledningsvist skal afholde en workshop for at afdække vores behov.\\
Et bud på funktionaliteter, vi søger, kunne være:
Benytte PDF som baggrundskort med GPS
Oprette registreringer med bl.a. lokalitet, billeder og tekst
Eksportere registreringerne til mail, Dropbox eller Onedrive.\\
Størstedelen af medarbejderne i afdelingen bruger IOS produkter, og vi vil derfor foretrække, at produktet blev
udviklet til denne platform, men dette er ikke et krav. \\
Rambøll stiller om nødvendigt en Apple Developer licens til rådighed.
Under projektet stiller Rambøll medarbejdere til rådighed til projektafgrænsningsmøder, workshops, interviews
eller lignede i det omfang, der er behov for det. 