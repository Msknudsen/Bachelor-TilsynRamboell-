\chapter{Udviklingsværktøjer}
Herunder vil de udviklingsværktøjer der er brugt under udviklingen af 
produktet blive beskrevet.

\subsection*{Git}
Git\cite{GitRef} er et versionsstyrings-værktøj der i dette projekt er 
blevet brugt til at holde styr på backup af samtlige filer. Git har under 
udviklingen af produktet været med til at tillade flere udviklere at 
arbejde på samme projekt/dokument uden alt for mange problemer.

\subsection*{\LaTeX}
\LaTeX \cite{LatexRef} er et opmærkningssprog brugt til tekstformatering af 
dokumenter, hvor alt bliver skrevet i plain-tekst. Under udviklingen af 
TrafficControl er LaTeX blevet brugt til alle dokumenter som 
møde-referater, dokumentationen m.m. To af grundende til brug af LaTeX i 
projektet er at det fungerer sammen med Git og at det tillader flere 
udviklere at arbejde parallelt i samme dokument.

\subsection*{Draw.io}
Draw.io\cite{Draw.io} er et online værktøj til udvikling af diagrammer. 
Det er enkelt at anvende og mængden af figurer er omfattende, hvilket medfører at en del af diagrammerne udviklet i forbindelse med Traffic Control er tegnet i draw.io.

\subsection*{Microsoft Visual Studio}
Visual Studio\cite{VisualStudio} er et integreret udviklingsmiljø fra 
Microsoft. Dette kan 	bruges til at udvikle alt fra konsolbaserede 
applikationer, GUI'er i WPF til hjemmesider. Er i projektet anvendt til at 
udvikle både Android-applikationen, webapplikationen samt alt backend.

\subsection*{Postman} 
Postman\cite{Postman} er en REST client som kører som applikation inde i 
webbrowseren Chrome. Den er anvendt til at teste TrafficControl API'et, da 
det er nemt at teste de forskellige REST funktionaliteter, GET, PUT, POST og 
DELETE.
\subsection*{Zenhub}
Zenhub \cite{Zenhub} er en process værktøj til Scrum udvikling.

\subsection*{Doxygen}
Doxygen \cite{DoxygenInfo} Bruges til at genere dokumentation fra koden. Doxygen blev brugt i data access layer.   

\subsection*{Anvendte frameworks} 
Et framework (eller programmeringsplatform) er en betegnelse for det miljø 
et program laves til at kunne udføres i. I udviklingen af Rambøll Tilsyns app
er flere frameworks anvendt. De forskellige frameworks beskrives i dokumentationen.
\begin{itemize}[-]
	\item Xamarin \cite{XamarinDoc}
	\item Xamarin.Firebase \cite{FirebaseDoc}
	
	%\item RestSharper \cite{RestSharp}
	%\item NUnit \cite{NUnit}
	%\item NSubstitute \cite{NSubstitute}
	
\end{itemize}
