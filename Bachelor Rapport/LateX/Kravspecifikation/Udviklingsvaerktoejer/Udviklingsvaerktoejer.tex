\chapter{Udviklingsværktøjer} \label{sec:Udviklingsvaerktoejer}
Herunder vil udviklingsværktøjer, der er brugt under udviklingen af 
produktet, blive beskrevet.

\subsection*{Git}
Git\cite{GitRef} er et versionsstyrings-værktøj, der i dette projekt er 
blevet brugt til at holde styr på versions historik af projektet. Git har under 
udviklingen af produktet været med til at tillade flere udviklere at 
arbejde på samme projekt/dokument.

\subsection*{\LaTeX}
\LaTeX \cite{LatexRef} er et opmærkningssprog brugt til tekstformatering af 
dokumenter, hvor alt bliver skrevet i plain-tekst. Under udviklingen af 
Rambøll Tilsyns App er LaTeX blevet brugt til alle dokumenter som 
mødereferater, de forskellige dokumentations dokumenter m.m. To af grundende til brug af LaTeX i 
projektet er, at det fungerer sammen med Git og at det tillader flere 
udviklere at arbejde parallelt i samme dokument.

\subsection*{Draw.io}
Draw.io\cite{Draw.io} er et online værktøj til udvikling af diagrammer. 
Det er enkelt at anvende og der er en stor mængde figurer at vælge i mellem. Dette medfører, at en del af diagrammerne udviklet i forbindelse med Rambøll Tilsyns App er tegnet i draw.io.

\subsection*{Microsoft Visual Studio}
Visual Studio\cite{VisualStudio} er et integreret udviklingsmiljø fra 
Microsoft. Dette kan bruges til at udvikle alt fra konsolbaserede 
applikationer, GUI'er i WPF til hjemmesider. Er i projektet anvendt til at 
udvikle både Rambøll Tilsyns App og Back-end.

\subsection*{Zenhub}
Zenhub \cite{Zenhub} er et procesværktøj til Scrum udvikling.

\subsection*{Doxygen}
Doxygen \cite{DoxygenInfo} bruges til at genere dokumentation fra soruce koden. Doxygen blev brugt i data access layer.   

\clearpage

\subsection*{Anvendte frameworks} 
Et framework (eller programmeringsplatform) er en betegnelse for det miljø, 
et program skrives i. I udviklingen af Rambøll Tilsyns app
er flere frameworks anvendt.
\begin{itemize}[-]
	\item Xamarin \cite{XamarinDoc}
	\item Xamarin.Firebase \cite{FirebaseDoc}
	
	%\item RestSharper \cite{RestSharp}
	%\item NUnit \cite{NUnit}
	%\item NSubstitute \cite{NSubstitute}
	
\end{itemize}
